\chapter{Spikes}
\label{sec:Spikes}

If a neuroscientist is asked by a layman about what we have learned about brain networks from electrical recordings, there is a high chance she first highlights
the studies of Hubel and Wiesel in the 1950s and 60s of neural representations in the primary visual cortex. In their pioneering studies, they measured spikes, the extracellular signatures of action potentials, of cortical cells by use of sharp recording electrodes. They found, for example, that many of the cells responded most vigorously, that is, with the highest number of spikes, to bar-like stimuli oriented in specific directions~\cite**{Hubel1959}. Later, the same approach has been used throughout the nervous system to map out how different neurons encode information on sensory stimuli, objects, spatial positions and more. Thus, the spike has arguably been the most important brain signal in systems neuroscience.

%%%%%%%%%%
% Figure: Intracellular and extracellular action potentials
%%%%%%%%%%
%\begin{cnfigure}{Figures/mm/EP-spike-Henze-w100-r150}
\begin{figure}[!ht]
\begin{center}
\includegraphics{Figures/Spikes/Spikes-Henze-w100-r150}
\end{center}
\caption[Intracellular and extracellular action potentials]{\textbf{Simultaneous recording of intracellular and extracellular action potentials (`spikes').}
Adapted from \citeasnoun**{Henze2000}.
\gen{Figure + caption to be updated}
}
\label{Spikes:fig:Henze}
\end{figure}
%%%%%%%%%%%

While the amplitude of intracellular the action potential is about 100~mV, the amplitudes of spikes are typically less than 1~mV. 
Also, the shape of the spike is very different from the shape of intracellularly recorded action potentials (Fig.~\ref{Spikes:fig:Henze}).
When detecting spikes, the main issue is that the spike amplitude is larger than the ambient noise level, typically \gex{XX}~$\mu$V. 
\gen{Har vi et godt tall med medhoerende referanse her?} 
\ehnote{Tja. Ambient noise er jo veldig avhengig av forskjellige faktorer, elektrode-type og impedans, in vivo vs. in vitro, vaaken vs. bedoevet tilstand osv, valg av filter, osv.}
\tvnnote{Alessio sa de typisk ikke tar med spikes med amplitude under 30 uV, og det var dette vi brukte i Buccino et al., 2018, J Neurophysiol, men han hadde ikke noe god kilde ellers}
For the detection of spikes, the detailed spike shape is of less importance. However, an extracellular electrode will in general measure spikes from several neurons positioned in the vicinity of the electrode contact. To obtain spike trains from individual neurons, the recorded spikes must thus be sorted in a process referred to as spike sorting~\cite**{Quiroga2007}. And in this process, the differences in spike shapes are crucial~\cite**{Einevoll2012}. 
Spikes shapes are also used to distinguish spikes from different neuron types. The temporal width of the spike is, for example, used to separate putative inhibitory neurons (narrow spikes) from excitatory neurons (broad spikes). But a more detailed separation into subgroups is also possible~\cite**{Buccino2018}.

Detailed modeling of spikes are important for several reasons: for example, 
(i) to understand what the single-neuron properties can tell us about spike shapes (and the other way around) \cite**{Holt1999,Gold2006,Pettersen2008a}
(ii) to estimate biases regarding which types of neurons preferentially generate the spikes recorded by electrodes, 
and 
(iii) to provide benchmarking data for development and testing of methods for spike sorting or spike-based neuron identification \cite**{Einevoll2012,CamunasMesa2013,Hagen2015,MondragonGonzalez2017,Buccino2020}

%%%
\section{\orange{Recording of spikes}}
\label{Spikes:sec:recording}
Spike recordings are typically done in living brains. Here the signal is obtained by high-pass filtering the extracellular potentials, with a lower cut-off set at 500~Hz or so.
The low-frequency part, the local field potential (LFP), is thought to mostly reflect synaptic inputs onto populations neurons around the contact~\cite**{Einevoll2013}. 

A common misunderstanding is that a spike only consists of high-frequency components.
The reasoning is that as since it lasts only a couple of milliseconds and an oscillation with a period of, say, 
2 milliseconds corresponds to a frequency of 500 hertz, lower frequencies will be absent or at least very weak.
But this is not generally true~\cite**{Pettersen2008,Schomburg2012,Ray2011,Scheffer-Teixeira2013}. 
Even a fast sodium spike contain frequency components with frequencies as low as 100 Hz \cite**{Pettersen2008a}.
This is illustrated in Figure~\ref{Spikes:fig:freq_dep} showing that also very narrow spikes may contain low-frequency components.
In fact, a so-called $\delta$-function pulse which essentially has zero width, is built up of equal amounts of frequency components for all frequencies.
And slower phenomena such as calcium spikes~\cite**{Stuart2007} and spike afterhyperpolarization \cite**{Buzsaki1988} further contribute to low-frequency components~\cite**{Buzsaki2012}. 

%%%%%%%%%
% Figure
%%%%%%%%%
\begin{figure}[!ht]
\begin{center}
%\includegraphics[width=0.6\textwidth]{Figures/Spikes/Spikes-eap_illustration.png}
\includegraphics[width=1.0\textwidth]{Figures/Spikes/Spikes-fig_spike_freq_content_amp.png}
\end{center}
\caption{\textbf{Spike and frequency components}}
\label{Spikes:fig:freq_dep}
\end{figure}
%%%%%

%%%%%%%%%
% Figure
%%%%%%%%%
\begin{figure}[!ht]
\begin{center}
%\includegraphics[width=0.6\textwidth]{Figures/Spikes/Spikes-eap_illustration.png}
\includegraphics[width=1.0\textwidth]{Figures/Spikes/Spikes-fig_delta_pulse_freq_content.png}
\end{center}
\caption{\textbf{Delta pulses and frequency components}}
\label{Spikes:fig:freq_dep_delta}
\end{figure}
%%%





The high-pass filtering of the extracellular signal typically used in extraction of spikes from extracellular recordings, thus changes the shape and reduces the signal power of the extracellular spike signal. 
The rationale for this high-pass filtering is rather to isolate the spikes from the the signal by removing the LFP part. The LFP
signal typically dominates in terms of overall power, but has very little power above a few hundred hertz~\cite**{Pettersen2008,Einevoll2013}.

\gen{Boer etter eller annet sted si noe om status for maalinger av spikes a.la. beskrivelse i bokkapittel av Anastassiou og co i "Principles of Neural Coding (2013)"}

%%%
\section{\orange{Spikes from multi-compartment neuron models}}
\label{Spikes:sec:multi-compartment}
%%%
Unlike the intracellular somatic action potential which has a quite standardized appearance, the spike shapes and amplitudes can vary
strongly depending on the position at which it is recorded.  This is illustrated in Fig.~\ref{Spikes:fig:MultiCompartment} showing computed spikes at various 
positions around a neuron during the firing of an action potential. 
Here the scheme described in \gex{XX} is used with an biophysically detailed neuron model for a layer-5 pyramidal cell.
One striking features is the rapid decay of spike amplitude with distance from the soma. Another is ...
\gen{Skriver mer naar figur er klar}


%A more detailed picture of spike shapes is obtained by considering a detailed multi-compartmental neuron model
%with a comprehensive branching structure typical for real neurons as in Figure~\ref{Spikes:fig:MultiCompartment}.
%With this dendritic morphology the membrane currents through the dendrites are spread over a larger membrane area.
%As a result, Equation~\ref{XX:equation:Ve-multi-compartment} predicts that the largest EP spikes will be seen
%around the soma for the example pyramidal neuron in Figure~\ref{Spikes:fig:MultiCompartment}.  
%Around the apical dendrites, the spikes will still have an inverted shape compared to spikes close to the soma. 
%However, their amplitudes will be small, only a few microvolts, so they will not be seen in most experiments.
%
%As for the two-compartment spike model, the spike amplitude in Figure~\ref{Spikes:fig:MultiCompartment} 
%decays sharply with distance from the neuron. In addition, the spike width increases
%with distance as demonstrated by the insets at two example positions. For the large spike recorded next to
%the soma the half-width of the spike is $\sim$0.6~ms, while at the position outside the dendrite, the half-width is increased to
%$\sim$0.7~ms. This corresponds to a low-pass filtering in the sense that the distant EP has lost some 
%high-frequency components compared to the EP close to the soma. This filtering effect is absent for the spike generated by the 
%two-compartment neuron model, and reflects that the cable properties of dendrites are important in determining 
%also the shape of recorded spikes.   
%
%*******


 %%%%%%%%%%
% Figure: Spike from multi-compartment model
%%%%%%%%%%
%\begin{cnfigure}{Figures/mm/EP-spike-MultiCompartment-w100-r150}
\begin{figure}[!ht]
\begin{center}
\includegraphics[width=1.0\textwidth]{Figures/Spikes/Spikes-fig_hay_eap_simpler}
\end{center}
\caption[]{\textbf{EP (spike) from action potential for a multi-compartmental neuron model
\ehnote{Figure adapted from Fig4. i Hagen2015 kanskje? Aksonet peker i "feil" retning ogsaa. 
Det boer nevnes hva som driver depolariseringen, ser ut som en step current?}
\tvnnote{Vi ville ha en litt enklere versjon enn Fig. 4 i Hagen2015, men enig i at vi kanskje bør vurdere å fikse retningen til axonet. Har du koden for dette? Stimuli er foreløbig synaptisk input i soma}
}. 
%Active sodium
%and potassium conductances in the soma compartment and computed from 
%Equation~\ref{XX:equation:Ve-multi-compartment}.
%Left panel shows the EP at different spatial positions, while right panel shows the corresponding
%soma membrane potential during the action potential. 
%\gen{Figure + caption to be updated. Change to Hay-model,}
}
\label{Spikes:fig:MultiCompartment}
%\figpermOurs
\end{figure}

%%%
\section{\orange{Spikes from two-compartment neuron model}}
\label{Spikes:sec:two-compartment}
%%%
While intracellularly recorded action potentials  can be modelled with a single-compartment neuron model, the two-compartment model
model is the simplest neuron model that can provide an extracellular spike. An example of a spike generated by such a model is shown in
Figure~\ref{Spikes:fig:DifferentNeuronModels}A. Around the soma, characteristic
spikes with a sharp negative peak followed by a slower positive hump is seen, in accordance with typical experimental
spike recordings as exemplified in Figure~\ref{Spikes:fig:Henze}. The same spikes, though inverted, are observed outside
the dendrite compartment. Such inverted spikes are rarely, if ever, seen in experiments, however. This suggests that while the two-compartment model can account for large spikes recorded next to the soma, it is inadequate when for 
predictions of  the detailed spatial pattern of  spike shapes that would be recorded by an electrode at different positions around a neuron.        

\gen{Skriver mer naar figuren er klar:}
\begin{itemize}
\item Compare with multicompartmental model results in Figure~\ref{Spikes:fig:DifferentNeuronModels}(left)
\item This filtering effect is absent for the spike generated by the two-compartment neuron model, 
and reflects that the cable properties of dendrites are important in determining 
also the shape of recorded spikes.   
\item New analytical results from Klas? (Similar to what was done for ball-and-stick in \citeasnoun**{Pettersen2008a}) 
\gen{Maybe not, looked into it a bit myself, and not quite sure how illuminatings these are ...}
\end{itemize}


%%%%%%%%%%
% Figure: Spikes from two-compartment, ball-and-stick and multicompartment
%%%%%%%%%%
%\begin{cnfigure}{Figures/mm/EP-spike-TwoCompartment-w100-r150}
\begin{figure}[!ht]
\begin{center}
\includegraphics[width=1.2\textwidth]{Figures/Spikes/Spikes-compare_hay_bns_2c_reformat_2}
\end{center}
\caption{\textbf{EP (spike) from action potential for a two-compartment neuron model (left), ball-and-stick-model (center),
multicompartment model (right).}
Her kunne vi kanskje bruke passive dendritter og clampe soma-spenningen til en bestemt form slik at resultatene fra de
tre modellene blir mer sammenlignbare?
}
\label{Spikes:fig:DifferentNeuronModels}
\end{figure}

%%%
\section{\orange{Spikes from ball-and-stick neuron}}
\label{Spikes:sec:ball-and-stick}

\gen{Note that this topic was discussed in more detail in Pettersen2012. Maybe expand present text? Will consider
after new figures have been made.}

Just like for people, some neurons are louder than others and generate spikes with larger
amplitudes.Thus when a recording electrode is lowered into the brain, there will be a bias regarding which
neurons will be recorded from, that is, the sampling will be skewed towards the large-spike neurons.
This bias should ideally be considered when analysing the data.

So what is the link between the morphology and electrical properties of neurons and the shape and size of the spike it produces and action potential. To explore this link \citeasnoun**{Pettersen2008a} considered 
ball-and-stick neurons where a dendrite cable `stick' is connected to a point-like soma.
While simple, the spikes of the ball-and-stick neuron model exhibits key qualitative features 
observed for biophysically detailed multicompartmental models, 
see Figure~\ref{Spikes:fig:DifferentNeuronModels}.  In particular, it exhibits 
rapid attenuation of spike amplitude  and increased spike width as the distance from the soma increases.

%%%%%%%%%%
% Figure: Spike widths and amplitudes
%%%%%%%%%%
%\begin{cnfigure}{Figures/mm/EP-spike-ball-and-stick-results-w100-r150}
\begin{figure}[!ht]
\begin{center}
\includegraphics{Figures/Spikes/Spikes-ball-and-stick-results-w100-r150}
\end{center}
\caption[]{
Spike widths (left) and (peak-to-peak) spike amplitudes (right) as a function of
distance from soma for a detailed pyramidal cell model (pyramidal) and two
types of ball-and-stick models: long, finite ball-and-stick
model (fin.~stick, long) with diameter $d=2~\mu$m and length
$l=1$~mm and a short, finite ball-and-stick model (fin.~stick, short) with diameter $d=1~\mu$m and length $l=0.2$~mm,
see Figure~\ref{Spikes:fig:ball-and-stick-neuron-models}.
The intracellular action potential shown in the inset in the left panel of 
Figure~\ref{Spikes:fig:ball-and-stick-frequency} was imposed as a voltage-clamp in the soma.
The EP was recorded in the
somatic plane normal to the stick/primary apical dendrite. 
In right panels guidelines illustrating the power-law decays $1/r$ and
$1/r^{2}$ have been added. 
For further details see \citeasnoun[Figure 6]{Pettersen2008}.
\gen{Figure + caption to be updated.} 
Adapted from \citeasnoun**{Pettersen2008}. 
\gen{Her kunne man jo kjoert ut nye resultater for Hay-modellen, men det er vel like greit aa bare bruke figurene fra 
Pettersen2008?} \tvnnote{Gaar jo forsaavidt fort aa lage paa nytt, og kan jo vaere fint at det blir en del av koden leserne har tilgjengelig?}
}
\label{Spikes:fig:ball-and-stick-results}
%\figpermOurs
\end{figure}
%%%%

Figure~\ref{Spikes:fig:ball-and-stick-results} shows the distance dependence of these spike measures both for a detailed multi-compartmental pyramidal neuron model and ball-and-stick neurons.
While the spike width and amplitude of the multi-compartmental neuron are larger than for
the two example ball-and-stick neurons (with short and long dendrite sticks, respectively), 
the shapes of the curves are similar. \ghnote{Kan vi ikke velge parametere slik at amplituden og shapen ligner i de to tilfellene?} 
\gen{Jo, det kan sikkert vaere mulig. Dette er klippet rett fra Pettersen2008a.}
Note also that the results for a ball-and-stick neuron with an infinitely long dendritic stick is
effectively identical to the long-stick results in the figure~\cite**{Pettersen2008}.

%
%%%%%%%%%%%
%% Figure: Neuron models considered in plot of spike widths and amplitudes
%%%%%%%%%%% 
%%\begin{cnfigure}{Figures/mm/EP-spike-ball-and-stick-neuron-models-w43-r300}
%\begin{figure}[!ht]
%\begin{center}
%\includegraphics{Figures/Spikes/Spikes-ball-and-stick-neuron-models-w43-r300}
%\end{center}
%\caption[]{\textbf{Neuron models considered in results in Figure~\ref{Spikes:fig:ball-and-stick-results}}. 
%\gen{Mulig vi ikke trenger en slik figur lenger pga Fig 8.3}
%Adapted from \citeasnoun**{Pettersen2008}.
%}
%\label{Spikes:fig:ball-and-stick-neuron-models}
%%\figpermOurs
%\end{figure}

\subsection{\orange{Frequency components of spikes}}
\label{Spikes:sec:Fourier}

To understand the physical origin of these results it is easier to consider each frequency 
component of the action potential separately. From Fourier theory it follows that any signal
such as the time course of a spike $V_\mathrm{e}(t)$, can be written as a sum of waves with different frequencies.
Such a Fourier sum can be constructed in various ways. The derivations presented below 
building on \citeasnoun**{Pettersen2008}, use the convention that a time signal $S(t)$ can be written as
%%%
\begin{equation}
S(t) = Re \{ \sum_{f}  \hat{S}(f) \exp (j 2 \pi f t) \}
\label{Spikes:equation:Fourier_sum}
\end{equation}
%%%
where $Re\{z\}$ represents the real part of the complex number $z$.  
Here $j$ is the unit of imaginary numbers, $f$ is the frequency, and the weight $\hat{S}(f)$ is in general  
complex. 

Figure~\ref{Spikes:fig:ball-and-stick-frequency} illustrates the Fourier decomposition of the
intracellular action potential (membrane potential) and extracellular spike, respectively.
In panel B the weight of the different frequency components needed to represent the signals are shown.
A key observation here is that unlike for the intracellular action potential, the largest
contributions to the extracellular spike, comes from frequencies larger than 100~hertz.
\ghnote{Det overrasket at f-spektrum til intra vs ekstra er saa fundamentalt ulike. Er signalet i Figurpanel A ufiltrert, mens det i B er filtrert?}
\gen{Nei, dette skyldes fysikken. Extra reflekterer transmembrane stroemmer som inneholder mer av de hoeye frekvensene enn det "lav-frekvente" membran-potensialet.}
\tvnnote{De er forøvrig betydelig likere i mine simuleringer med Hay modellen, selv om det generelle mønsteret vel kanskje er omtrent det samme som her.}


%%%%%%%%%%
% Figure: Action potential and its frequency content
%%%%%%%%%%
%\begin{cnfigure}{Figures/mm/EP-spike-ball-and-stick-frequency-w90-r150}
\begin{figure}[!ht]
\begin{center}
\includegraphics{Figures/Spikes/Spikes-ball-and-stick-frequency-w90-r150}
\end{center}
\caption[]{(a) Action potential used in simulations in Figure~\ref{Spikes:fig:ball-and-stick-results}(inset) 
and its frequency content.
%The intracellular spike width is defined as the
%width of the AP at half amplitude and is 0.55~ms for the standard
%AP, and half the value for the narrow AP. 
(b) Frequency content of example extracellular spike.
Inset: Typical spike shape computed a distance $r=10~\mu$m
perpendicular to the dendrite at the level of soma for a ball-and-stick 
neuron with diameter $d=2~\mu$m and infinite dendrite length.
Here the intracellular action potential in left panel was imposed as a voltage-clamp in the soma.
The extracellular spike width is defined as the width of the negative phase at 25\% of its maximum
amplitude and is 0.44~ms for the example spike.
The amplitude (peak-to-peak value) of the spike is $56~\mu$V  
\gen{Figure + caption to be updated. Will only include standard AP in final figure.} 
Adapted from \citeasnoun**{Pettersen2008}.}
\gen{Her kunne man jo kjoert ut nye resultater for Hay-modellen, men det er vel like greit aa bare bruke figurene fra 
Pettersen2008?}
\label{Spikes:fig:ball-and-stick-frequency}
%\figpermOurs
\end{figure}

%\subsection{Frequency-dependent electrotonic length constant}
%\label{Spikes:sec:Fourier}
%
%For the ball-and-stick neuron, a membrane current entering the soma has to return to ECS through the cable stick, see Figure~\ref{Spikes:fig:ball-and-stick-sketch}. When an oscillating membrane current is entered through the soma, the spatial pattern of return current will depend on the frequency of the oscillation due to the capacitive properties of the membrane: For higher frequencies the capacitive membrane current will be larger, and the 
%membrane effectively more leaky. Thus the injected soma current will return closer to the soma for higher frequencies, 
%as  seen in the inset in Figure~\ref{Spikes:fig:ball-and-stick-sketch}.  

\subsection{\orange{Approximate spike formulas}}
\label{Spikes:sec:approximate}

To compute the EP generated by an imposed oscillatory current, the stick must be divided into small compartments  
and the multi-compartmental EP formula in Equation~\gex{XX} used. 
This in general requires detailed knowledge of the resulting membrane currents in all compartments. 
While these can be computed for the ball-and-stick neuron (see, for example, \citeasnoun**{Pettersen2014}), 
the resulting formula for the EP becomes cumbersome and difficult to interpret.  
However, some useful insights can be obtained in two limiting cases: the recording being very near to the soma 
or very far from the soma.

%For recording positions very close to the soma, the contribution from the soma current will dominate 
%in the sum in Equation~\gex(XX). In this case the distance from the electrode
%to the return currents in the dendrite will be much larger than the distance to soma. 
%
%%
%With only the contribution from
%the soma compartment included, the amplitude $|\hat{V}_\mathrm{e,near}(f,\vec{r})|$  
%of the EP signal for each Fourier component with frequency $f$ is found to be

\subsubsection{Spikes recorded near the soma}
\label{Spikes:sec:near-spikes}
\tvnnote{En del bruker den deriverte av membranpotensial til å approksimere spikes. Bør det taes med?}
To understand the dependence of spike shapes and amplitudes on model parameters for the ball-and-stick neuron, it is useful
to consider the extracellular potentials set up by each of the frequency (Fourier) components of the action potentials separately. 
For recording positions $\vec{r}$ very close to the soma, the contribution from the soma current will dominate over the contributions from the dendrite 
in the sum giving the EP in Equation~\gex{XX}. Then the amplitude of the
predicted oscillating EP $|\hat{V}_\mathrm{e,near}(f,\vec{r})|$ is approximately given by 
%
\begin{equation}
  |\hat{V}_\mathrm{e,near}(f,\vec{r})| = \frac{|\hat{I}_\mathrm{s}(f)|}{4 \pi \sigma r} 
  \label{Spikes:equation:Ve_near_1}
\end{equation}
%
where $|\hat{I}_\mathrm{s}(f)|$ is the amplitude of the oscillating current through the soma membrane.
%and also the axial current entering the dendrite from the soma compartment. 
For the relatively high frequencies of most relevance for the spike, this soma current is related to the soma membrane potential 
$\hat{V}_\mathrm{s}(f)$ through~\cite**{Pettersen2008}
%
\begin{equation}
  |\hat{I}_\mathrm{s}(f)| =  \frac{\pi^{3/2} d^{3/2}}{\sqrt{2}} \sqrt{ \frac{f C_\mathrm{m}}{R_\mathrm{a}} }  |\hat{V}_\mathrm{s}(f)|
  \label{Spikes:equation:Isoma}
\end{equation}
%
and the spike EP is thus found to be
%
\begin{equation}
  |\hat{V}_\mathrm{e,near}(f,\vec{r})| 
  = \frac{\sqrt{\pi}}{4 \sqrt{2} \sigma}
     \frac{d^{3/2}}{r} 
     \sqrt{ \frac{f C_\mathrm{m}}{R_\mathrm{a}} }  |\hat{V}_\mathrm{s}(f)| 
  \propto \frac{d^{3/2}}{r} \sqrt{ \frac{f C_\mathrm{m}}{R_\mathrm{a}} }  |\hat{V}_\mathrm{s}(f)| 
  \label{Spikes:equation:Ve_near_2}
\end{equation}
%
%%%%%%%%%%
%However, as shown in XX
%Box~\ref{mm:box:ball-and-stick-spikes} 
%some useful insights can be obtained in two limiting cases: the recording being very near to the soma or very far from the soma.

%For recording positions very close to the soma, the contribution from the soma current will dominate 
%in the sum in Equation~\ref{XX:equation:Ve-multi-compartment}. In this case the distance from the electrode
%to the return currents in the dendrite will be much larger than the distance to soma. With only the contribution from
%the soma compartment included, the amplitude $|\hat{V}_\mathrm{e,near}(f,\vec{r})|$  
%of the EP signal for each Fourier component with frequency $f$ is found to be
%%
%\begin{equation}
%  |\hat{V}_\mathrm{e,near}(f,\vec{r})| 
%  \propto \frac{d^{3/2}}{r} \sqrt{ \frac{f C_\mathrm{m}}{R_\mathrm{a}} }  |\hat{V}_\mathrm{s}(f)| 
%  \label{Spikes:equation:Ve_near}
%\end{equation}
%%
The suffix `near' is added because this expression only applies in the `near-field' limit, that is,
close to the soma.
$|\hat{V}_\mathrm{s}(f)|$ is the amplitude of Fourier component of the soma membrane potential at the same
frequency (see Figure~\ref{Spikes:fig:ball-and-stick-frequency}). Further, $d$ is the diameter of the dendritic
stick, $C_\mathrm{m}$ is the specific membrane capacitance, and $R_\mathrm{a}$ is the specific axial resistance.  

%One of the predictions from the formula in Equation~\ref{Spikes:equation:Ve_near_2} is that the amplitude of each Fourier component decays
%as $1/r$ when moving away from soma (yet staying within distances where the `near-field' approximation is still applicable).
%Since this applies to all Fourier components which together constitute the action potential, this implies that the amplitude of the 
%spike will decay as $1/r$ in this regime as well.
%\gen{From David Sterratt: Intuition for the relationship witd $d, C_m, f, r_a$.}

%%%%%%%%%%
% Figure: Frequency-dependent distribution of return currents
%%%%%%%%%%
%\begin{cnfigure}{Figures/mm/EP-spike-ball-and-stick-sketch-w70-r300}
\begin{figure}[!ht]
\begin{center}
\includegraphics{Figures/Spikes/Spikes-ball-and-stick-sketch-w70-r150}
\end{center}
\caption[]{Illustration of ball-and-stick neuron and its frequency-dependent 
distribution of dendritic return currents following injection of a sinusoidal current into the soma.
The net current entering the soma will enter the dendrite as an axial current, and return to the 
ECS via the dendrite membrane. The inset shows the spatial distribution of this return current
for different frequencies. The higher the frequency, the closer the return currents will be and
the smaller the frequency-dependent length constant $\hat{\lambda}(f)$, reflecting the weighted mean 
of the return-current positions (see sidebox), will be. 
\gen{Figure + caption to be updated.} Adapted from \citeasnoun**{Pettersen2012}.
}
\label{Spikes:fig:ball-and-stick-sketch}
%\figpermOurs
\end{figure}
%


\subsubsection{Spikes recorded far away from soma}
\label{sec:Spikes:far-spikes}

For spikes recorded close to the soma, a single membrane current, the soma current,
dominates the sum in Equation~\gex{XX}. 
For recording positions far away from the soma, the contribution from return membrane currents must also be included in the 
account in the sum. For the ball-and-stick neuron, a membrane current entering the soma has to return to ECS through the cable stick, see Figure~\ref{Spikes:fig:ball-and-stick-sketch}. When an oscillating membrane current is entered through the soma, the spatial pattern of return current will depend on the frequency of the oscillation due to the capacitive properties of the membrane: For higher frequencies the capacitive membrane current will be larger, and the membrane effectively more leaky. Thus the injected soma current will return closer to the soma for higher frequencies, as  seen in the inset in Figure~\ref{Spikes:fig:ball-and-stick-sketch}.  

An approximate way of including the contribution to the spike from the dendritic return currents is 
assume all return currents to leave the dendrite at a single point on the dendrite. Then we are left with
a current dipole where the transmembrane current entering at the soma is balanced with a oppositely directed current
with the same magnitude leaving at a single point on the dendrite. The current dipole length is then given by the distance
between the soma and the dendritic position of the return current. 

As shown in Figure~\ref{Spikes:fig:ball-and-stick-sketch} this current dipole length will depend on frequency.
A natural choice is to set it equal to the frequency-dependent length constant $\lambda_\mathrm{AC}(f)$ of the stick,
where the length constant corresponds is the mean value of the 
envelope of the sinusoidally varying (normalized) membrane current
$\hat{i}_\mathrm{m}$ weighted with distance $z$ from soma, see Figure~\ref{Spikes:fig:ball-and-stick-sketch}. 
For an infinite dendrite stick this corresponds to
%
\begin{equation}
  \lambda_\mathrm{AC}^\infty(f) = \frac{\int_0^\infty z |\hat{i}_\mathrm{m}| dz}{\int_0^\infty |\hat{i}_\mathrm{m}| dz} 
  \label{Spikes:equation:formula_lambda_ac}
%=  \frac{\sqrt{2}\lambda}{\sqrt{\sqrt{W^2+1}+1}}.
\end{equation}
%
For high frequencies ($f \gg 1/2 \pi R_\mathrm{m} C_\mathrm{m}$) this is after some algebra 
found to give (see \citeasnoun[Appendix C]{Pettersen2008} for details)
%
\begin{equation}
 \lambda_\mathrm{AC}^\infty(f) =  \frac{\lambda}{\sqrt{\pi f \tau}} = 
  \frac{1}{2\sqrt{\pi}} \sqrt{\frac{d}{f R_\mathrm{m} C_\mathrm{m}}}
\label{Spikes:equation:approx_lambda_ac}
\end{equation}
%
where $\lambda$ is the cable length constant from Equation~\gex{XX}.


Then the extracellular potential contribution can be approximated by using the 
dipolar expression in Equation~\gex{XX}, that is,
%%%
\begin{equation}
  |\hat{V}_\mathrm{e,far}(f,\vec{r})| =  \frac{|p(f) \cos \theta|}{4 \pi \sigma r^2} 
                                            = \frac{| \hat{I}_{s}(f) \lambda_\mathrm{AC}(f) \cos \theta|}{4 \pi \sigma r^2}   
                                                                                        \label{Spikes:equation:Ve_far_1}
\end{equation}
%Chapter~\ref{XX:chap:XX}

%
Thus for EPs measured far away from the soma we find 
%  
\begin{equation}
  |\hat{V}_\mathrm{e,far}(f,\vec{r})|  = \frac{1}{8 \sqrt{2} \sigma} d^{2} \frac{1}{r^2  R_\mathrm{a}} 
      |\hat{V}_\mathrm{s}(f) \cos \theta | 
  \propto d^{2} \frac{|\cos \theta|}{r^2  R_\mathrm{a}} |\hat{V}_\mathrm{s}(f)| 
  \label{Spikes:equation:Ve_far_2}
\end{equation}
The suffix `far' is added because this expression only applies in the `far-field' limit, that is,
far away from the soma. 
%
%Equations~(\ref{Spikes:equation:Ve_near_2}) and (\ref{Spikes:box:equation:Ve_far_2}) describe how each frequency component of 
%the soma membrane potential  $\hat{V}_\mathrm{s}(f)$ is `translated' into frequency components of the 
%EP spike ($\hat{V}_\mathrm{e,near}(f,\vec{r})$ and $\hat{V}_\mathrm{e,far}(f,\vec{r})$, respectively). 
%%\end{boxfloat}
%%%%%%%%%%%%%%%%%%%%%%%%%%%%%%%%%%%%%%%%%


%An estimate of this length is provided by the 
%frequency-dependent length constant  $\lambda_\mathrm{AC}(f)$  corresponding to the weighted mean of the positions of the return currents along
%the dendrite stick (see Box XX). 
%Then with the use of expression for the EP around a current dipole in 
%Equation~\ref{XX:equation:Ve-dipole-p}:
%%%%
%\begin{equation}
%  |\hat{V}_\mathrm{e,far}(f,\vec{r})|  \propto d^{2} \frac{|\cos \theta| }{r^2  R_\mathrm{a}}  |\hat{V}_\mathrm{s}(f)| 
%  \label{Spikes:equation:Ve_far}
%\end{equation}
%%%%



%A first observation from this formula is that the EP is no longer radially symmetric, and depends both on the radial distance $r$ from the neuron and  the angle $\theta$ with the dipole axis, that is, the direction of the dendritic stick.
%The amplitude will be largest above and below the neuron where $\theta=0^\circ$ and $\theta=180^\circ$, respectively. In the sideways direction 
%($\theta \sim 90^\circ$) the EP will be much smaller, as is characteristic for spatial pattern of potentials around a current
%dipole as illustrated in Figure~\ref{Spikes:fig:TwoCompartment}. A qualitatively similar dipolar pattern, although not so distinct,
%is also seen for the spike generated by the biophysically detailed multi-compartment neuron in Figure~\ref{Spikes:fig:DifferentNeuronModels}.


%Another difference of this far-field expression with the near-field expression in Equation~\ref{Spikes:equation:Ve_near}, is that the
%amplitude decays as $1/r^2$, characteristic for potentials around dipolar sources, rather than $1/r$ which is characteristic for potentials around 
%a single source. This transition from a $1/r$ `monopolar' regime to a $1/r^2$ dipolar regime is indeed observed in
%the spike-amplitude panel in Figure~\ref{Spikes:fig:ball-and-stick-results}.

\subsection{\orange{Spike amplitude dependence on distance}}
A prediction from the near-field formula in Equation~\ref{Spikes:equation:Ve_near_2} is that the amplitude of each Fourier component decays as $1/r$ when moving away from the soma, whilst staying within distances where the `near-field' approximation still applies.
Since this applies to all frequency components which together constitute the action potential, this implies that the amplitude of the 
spike will decay as $1/r$ in this regime as well. 

Far away, the far-field formula (Equation~\ref{Spikes:equation:Ve_far_2}) implies that the spike amplitude depends not only depend on the radial distance $r$ from the neuron, but also the angle $\theta$ with the dipole axis, the direction of the dendritic stick. The amplitude will be largest above and below the neuron where $\theta=0^\circ$ and $\theta=180^\circ$, respectively. In the sideways direction 
($\theta \sim 90^\circ$) the spike will be much smaller, as is characteristic for spatial pattern of potentials around a current
dipole as illustrated in Figure~\ref{Spikes:fig:DifferentNeuronModels}. A qualitatively similar dipolar pattern, although not so distinct,
is also seen for the spike generated by the biophysically detailed multi-compartment neuron in 
Figure~\ref{Spikes:fig:DifferentNeuronModels}.
Another difference of this far-field expression with the near-field expression in Equation~\ref{Spikes:equation:Ve_near_2}, is that the
amplitude decays as $1/r^2$, characteristic for potentials around dipolar sources, rather than $1/r$ which is characteristic for potentials around 
a single source. %This transition from a $1/r$ `monopolar' regime to a $1/r^2$ dipolar regime was also found in model studies with biophysically detailed 
%neurons \cite[Fig.~X]{Pettersen2008}.
%\todo{Connect to Figure :EP-spike-ball-and-stick-results}

An overall observation is that the spike is quite local, with the amplitude of the spike decaying rapidly with distance from the neuron soma. 
For the pyramidal neuron considered in Figure~XX, for example, the spike amplitude decays from about
300 microvolts a distance 20 micrometers from the soma centre to only about 10 microvolts a
distance 100 micrometers away. \gen{Update when new figures are added}
This rapid decay eases the interpretation of recorded spikes, since it implies
that in practice an electrode contact will only pick up spikes from neurons with somas positioned within a radius of some tens of micrometers.
%\todo{GTE: Rewrite this to compare with our own figure spikes around multicompartmental model + refer to Pettersen2008}  

%\paragraph{Spike amplitude dependence on neuronal parameters}
\subsection{\orange{Spike amplitude dependence on neuronal parameters}}
The spike amplitude is proportional to $d^{2}$ far away from the soma and to $d^{3/2}$ close to the soma,
where $d$ is the diameter of the dendritic stick diameter. This implies that far away from the soma the spike amplitude is proportional to  the cross-sectional area of the dendrite. Close to the soma, the spike amplitude also increases with dendrite diameter, but slightly less so,.

Another observation is that the spike amplitude is independent of the membrane resistance $R_\mathrm{m}$ of the dendrite.
This reflects that the frequencies dominating the spike are so high that the ionic membrane current, governed by $R_\mathrm{m}$, is
much smaller than the capacitive membrane current, governed by membrane capacitance $C_\mathrm{m}$.  
Thus the spatial distribution of the return current along the dendrite will depend only on the capacitive current. This dependence
is seen through the presence of  $C_\mathrm{m}$ in the near-field formula in Equation~\ref{Spikes:equation:Ve_far_2}. 
Note that in the far-field formula Equation~\ref{Spikes:equation:Ve_far_2},  $C_\mathrm{m}$ is absent due to cancellation with another factor containing $C_\mathrm{m}$ in the mathematical derivation, 
cf. \citeasnoun[Equation 23]{Pettersen2008}.)

In Equations~\ref{Spikes:equation:Ve_near_2} and \ref{Spikes:equation:Ve_far_2}, the spike amplitude is reduced when the 
axial resistance $R_\mathrm{a}$ in the dendrites is increased. This reflects that an increased axial resistance implies that the current entering
the soma during the first phase of an action potential will return closer to the soma. This implies shorter distances on average between the sink (soma) and the sources, where the current return to the ECS, and thus a smaller current dipole and a smaller spike.


\subsection{\orange{Spike shape dependence on distance}}
The near-field and far-field formulae in Equation~\ref{Spikes:equation:Ve_near_2} and Equation~\ref{Spikes:equation:Ve_far_2} respectively also give qualitative insights regarding the shape of the spike. In the near-field expression, the high-frequency components of the spike is amplified 
compared to the low-frequency components, with $\hat{V}_\mathrm{e,near}(f,r) \propto \sqrt{f}$.
Thus close to the soma the spike is observed to be sharper than the intracellular action potential,
as observed in the insets in Figure~\gex{XX}. \gen{If we include such in new figures.} 
In the far-field regime there is no such high-frequency amplification ($\hat{V}_\mathrm{e,near}(f,r) \propto f^0 \sim 1$).
As a consequence, spikes measured far away from the soma will have less high-frequency content than those measured close to soma.
Thus the far-away spikes will be blunter and have larger spike widths as seen in the spike-width panel of 
Figure~\ref{Spikes:fig:ball-and-stick-results}.
%\todo{This has also been seen in experiments, but I don't have a reference right off the bat}.

%\paragraph{Generalisation of findings to other neuron models}
\subsection{\orange{Generalisation of findings to other neuron morphologies}}
While the formulae above were derived for a neuron model with a single passive dendritic stick, similar expressions can be derived for 
more complicated neuron models where several passive sticks protrude from the soma, see \citeasnoun**{Pettersen2008}.
The main conclusions above hold also for these neuron models, in particular that spike widths always increase with distance and that
the amplitude of a spike is proportional to $d^{k}$ where $k\sim1.5-2$. For neurons with many dendrites attached to the 
soma, the contributions to the spike amplitude roughly add up. A simple rule of thumb is that a neuron's spike amplitude is 
roughly proportional to the sum of the cross-sectional areas for all dendrite branches attached directly onto the soma. Neurons with many thick dendritic branches attached to the soma will thus generate the largest spikes. See~\citeasnoun**{Pettersen2008} for further discussion.
\gen{Kunne eventuelt aa vise noen resultater fra Jorgens thesis her}
\tvnnote{Føler dette er litt lite testet enn så lenge til å ha med i bok? Må i minste fall gjennskape resultatene, men kan godt gjøre det om du vil :-)}


%\subsection{Insights from spike near- and far-field expressions}
%
%There are several qualitative insights regarding the sizes and widths of spikes that can 
%be found from the near-field and far-field formulae in Equations~\ref{Spikes:equation:Ve_near_2} and 
%\ref{Spikes:equation:Ve_far_2}. One relates directly to the shape of the spike:
%in the near-field expression, the high-frequency components of the spike is amplified 
%compared to the low-frequency components, that is, $\hat{V}_\mathrm{e,near}(f,r) \propto \sqrt{f}$.
%Thus close to the soma the spike is observed to be sharper than the intracellular action potential
%as observed in the insets in Figure~\ref{Spikes:fig:ball-and-stick-frequency}. 
%In the far-field regime there is no such high-frequency amplification ($\hat{V}_\mathrm{e,near}(f,r) \propto f^0 \sim 1$).
%As a consequence, spikes measured far away from the soma will have less high-frequency content than those measured close to soma.
%Thus the far-away spikes will be blunter and have larger spike widths as seen in the spike-width panel of 
%Figure~\ref{Spikes:fig:ball-and-stick-results}.
%
%The spike amplitude is proportional to $d^{2}$ far away from the soma and $d^{3/2}$ close to the soma,
%where $d$ is the diameter of the dendritic stick diameter.
%This implies that far-way from the soma the spike amplitude is proportional to the cross-sectional area of the dendrite.
%Close to the soma, the spike amplitude also increases with dendrite diameter, but not so prominently as far away.
%Another observation is that the spike amplitude is independent of the membrane resistance $R_\mathrm{m}$ of the dendrite; only the membrane 
%capacitance $C_\mathrm{m}$ and the axial resistance $R_\mathrm{a}$ matter, 
%This reflects that the frequencies dominating the spike are so high that the capacitive membrane current 
%(governed by $C_\mathrm{m}$) is much larger than the ionic membrane current  (governed by $R_\mathrm{m}$). 
%%\todo{DCS: I'm wondering if we need more details on frequency-dependent cable theory, beyond Eq. 5.11.}
%
%An overall observation is that the spike is quite local, that is, the amplitude of the spike decays rapidly with distance from the neuron soma. For the pyramidal neuron considered in 
%Figure~\ref{Spikes:fig:ball-and-stick-results}, for example, the spike amplitude decays from being about
%300 microvolts a distance 20 micrometers from the soma center to being only about 10 microvolts a
%distance 100 micrometers away. This rapid decay eases the interpretation of recorded spikes, since it implies
%that in practice an electrode contact will only pick up spikes from neurons with somas positioned within a radius of some tens of micrometers.
%
%\gen{Add tekst on generalizability to more complicated models: ball-and-star ,,,}

%%%%%%%%%%%
%% Box: Ball-and-stick model for spikes
%%%%%%%%%%%
%%\begin{boxfloat}{Ball-and-stick model for spikes}
%%  \label{mm:box:ball-and-stick-spikes}  
%\subsection{\red{Box: Ball-and-stick model for spikes}}
%\gen{This was a box in the Sterratt chapter}
%%
%For recording positions further away from the soma, the contribution from return membrane currents must be taken into
%account in the sum in Equation~\ref{XX:equation:Ve-multi-compartment}. 

\section{\orange{Spikes in neurons with active dendrites}} 

\gen{This text is adapted from Pettersen2012. Should we add something, for example, a figure?}

In the above investigation of the ball-and-stick neuron, the assumption of an electrically passive dendritic stick was
essential. This assumption made the problem of relating intracellular potentials recorded in the soma to extracellular potentials recorded outside the neuron \emph{linear} and essentially independent of the detailed shape of the intracellular action potential:
the shape of the intracellular action potential only affected the weight of each frequency component in the Fourier sum.
Thus the above analytical insights apply in principle to all intracellular action-potential waveforms.

However, real neurons have active conductances also in the dendrites \cite**{Stuart2007}, making 
the problem nonlinear. The assumption of independent frequency component then no longer hold, and 
instead one has to resort to numerical investigations using the general formula in Equation \gex{XX} where
all active conductances are included explicitly. 

Using this scheme, Gold and coworkers \cite**{Gold2006,Gold2007} performed thorough investigations of the extracellular signatures 
of spikes from pyramidal neurons in hippocampus CA1 where active dendritic conductances were included in the model.
Their results were in qualitative agreement with many of the observations seen above 
for neurons with passive dendrites: (i) the spike width was seen increase with distance from the soma (cf.~Fig.~5A in Ref.~\citeasnoun**{Gold2006}), (ii) the spike amplitude was seen to decay with distance from the soma with a power between 1 and 2 for distances less than 50 $\mu$m (cf.~Fig.~14 in 
Ref.~\citeasnoun**{Gold2006}), and (iii) the spike amplitude was also seen to change significantly to varying intracellular resistivity $R_i$ and capacitance $C_m$, but not so much to varying membrane resistivity \citeasnoun**{Gold2007}. They also found that extracellular waveforms provide tight constraints on some of the neuronal model parameters, suggesting that extracellular spikes could be useful for constraining compartmental models \citeasnoun**{Gold2007}.



\section{\orange{Axonal contributions to spikes}}

In the investigations of the spike generated by the ball-and-stick neuron above, the assumption was that the action potential
is generated by active conductances in the soma and that the spike is generated by a current dipole reflecting the distribution of return currents in the dendritic stick. However, in some neurons the action potential is initiated at by the axon initial segment (AIS) some distance away from the soma~\cite**{Goethals2020}. This will in turn ignite the full soma action potential. In this initial phase of the action potential there will thus be a current dipole where current enters the neuron at AIS and leave through the soma. In this scenario there will expectedly be a short-lasting somatic source prior to the longer-lasting somatic sink setting up the characteristic 
strong negative sodium peak. Model simulations have indeed confirmed that this is feasible and that a small and narrow 
positive peak can be seen prior to the negative sodium peak for recording close to the soma~\cite**{Telenczuk2018}.
Interestingly, detailed experimental studies of the shape and amplitude of extracellular spikes recorded simultaneously around the soma and along the axon has now become possible by means of 
high-density microelectrode arrays (HD-MEAs)~\cite**{Emmenegger2019}.
\gen{Comment on axon-representations in Hay-model and Mainen-model.}

\section{\orange{Effects of measurement device on spike recordings}}

In the above examples we have assumed an infinite volume conductor, that is, the extracellular conductivity
has been assumed to be the same everywhere. We have also employed
the point-electrode approximation, that is, the recording electrode has been assumed  
to record the potential at one particular position in space (Sec.~\ref{VC:sec:point-electrode}). Also it has been assumed
to faithfully record the extracellular potential without disturbing the potentials around the neuron
in any way.  As discussed in Sec.\ref{VC:sec:electrodes}, however, real recording devices will in general affect
the measured potentials in several ways.  

\subsection{\orange{Physical sizes of contacts and shafts of recording electrode}}
\label{sec:Spikes:electrode_size}
Most electrodes used for extracellular recordings inside the brain consists of electrode contacts made of
highly conductive materials embedded in electrically insulating electrode shaft. The amplitudes and shapes of 
recorded spikes are affected by the size of the electrode contacts, and as discussed in Sec.~\ref{VC:sec:disc-electrode}
this effect can be modelled by use of the disc-electrode approximation. In this approximation the spike potential is computed
by averaging results from using the point approximation across the surface of the electrode, and it is thus straightforward to implement.
 An example of its use
is provided by Fig.~\ref{Spikes:fig:electrode_size}) showing spikes recorded with circular electrodes of different radii.
For one, the spike amplitude is seen to be reduced with increasing contact sizes (panel B). The shape is also affected as the 
averaging of the potentials across the contact surface will reduce high-frequency components of the spike (panel C).
\gen{Et eller annet sted bor vi kanskje nevne at dette er effekt nummer to som gjoer spiken mindre skarp (i tillegg til effekten
av oekt avstand fra soma)}

%%%
\begin{figure}[!ht]
\begin{center}
\includegraphics[width=0.6\textwidth]{Figures/Spikes/Spikes-fig_elec_size_effect.pdf}
\end{center}
\caption[]{\textbf{Effect of electrode size.}Bigger recording electrodes can cause smaller and broader EAPs.  \ghnote{Flott. Kursivere $r$? Forklare simuleringen?}
\gen{Veldig fin figur, som kanskje kan passe enda bedre here enn i VC-kapitlet "spikes" kapitlet. Hva tenker dere?}}
\label{Spikes:fig:electrode_size}
\end{figure}
%%%

The insulation electrode shaft effectively act to shadow spikes from neurons placed on the 'non-contact' side of the electrode.
Detailed studies of this requires comprehensive numerical investigations by use of FEM modeling~\cite**{Mechler2011,Mechler2012,Buccino2019}.

\subsection{\orange{Microelectrode arrays (MEAs)}}

Spikes are not only recorded in living brains. In \emph{in vitro} recordings, small slices of excised brain tissue are
placed in suitably designed dishes where physiological properties of cells and networks can be probed in detail for hours.
In so-called \emph{microelectrode arrays (MEAs)}, the bottom of the device is covered by a grid of electrode
contacts that records electrical signals generated by the neurons above. The brain slice and the MEA are both
covered with a liquid, typically saline, to protect the cells and keep them alive for the duration of the experiment, see
Figure~\ref{Spikes:fig:MEA-setup}.

%%%
\begin{figure}[!ht]
\begin{center}
\includegraphics[width=0.7\textwidth]{Figures/Spikes/Spikes-MEA-1-w43-r300}
\end{center}
\caption[]{\textbf{MEA-setup.}
Figure is adapted from \citeasnoun**{Ness2015}.}
\label{Spikes:fig:MEA-setup}
\end{figure}
%%%

In MEAs the electrode contacts are embedded in an insulating glass plate with very 
low electrical conductivity, while the covering liquid typically has a higher electrical conductivity 
than the brain slice it covers. Thus the extracellular conductivity around the signal-generating neurons
will not be constant as assumed above, and this will affect the amplitude and shape of the recorded 
spikes. 

In general, FEM modeling will be required to solve the forward-modeling for situations such as this where
the the conductivity $\sigma$ varies with position. However, if we assume that the MEA substrate, slice and saline
all extend infinitely in the lateral directions so that $\sigma$ can be assumed to only have planar step-wise discontinuities, 
formulas analogous to Equation~\ref{XX:equation:Vr} can be derived by use of the
\emph{method of images} from electrostatics. \gen{Referer til section i VC.}


%%%
\begin{figure}[!ht]
\begin{center}
\includegraphics[width=0.8\textwidth]{Figures/Spikes/Spikes-MEA-2-w43-r300}
\end{center}
\caption[]{\textbf{MEA-spikes.}
Figure is adapted from \citeasnoun**{Ness2015}.}
\label{Spikes:fig:MEA-spikes}
\end{figure}
%%%

An example result is shown in Figure~\ref{Spikes:fig:MEA-spikes}.
The largest effect comes from the insulating glass substrate which roughly doubles the size of the 
recorded spikes. However, also the highly-conductive 
saline cover employed in the example has an effect. More specifically, it
reduces the size of the spike compared to the hypothetical 
situation where the saline had the same value of the electrical conductivity as the brain slice.


\section{\red{Multi-unit activity (MUA)}}

In general, a recording contact will pick up spikes from several neurons positioned in its vicinity, that is, from `multiple units'.
The term \emph{multi-unit activity (MUA)} refers to this total spiking activity.

%%%
\begin{figure}[!ht]
\begin{center}
\includegraphics[width=0.8\textwidth]{Figures/Spikes/MUA-11}
\end{center}
\caption[]{\textbf{MUA tetrode}
Fra Hagen (2015):
Example benchmarking data for in vivo tetrode recordings. (a) Extracellular potential generated by population of six cortical pyramidal neurons.  
(b) Raw benchmarking data found from superposition of population potentials (from panel a) and synthesized model noise. (c) Filtered benchmarking data (corresponding to signals in (b)) showing MUA". 
Adapted from \citeasnoun**{Hagen2015}.
}
\gen{Kan vi kan legge til et panel som illustrerer en tetrode?}
\label{fig:Spikes:MUA-tetrode}
\end{figure}
%%%


\subsection{\orange{Spike extraction}}  

With only a single recording contact, the most direct way to measure the MUA is to simply detect and count the spikes in the recorded 
signal trace. Here a suitable detection procedure must be used. This typically involves thresholding where only putative spikes larger than a 
preset threshold depending on the ambient noise level, are included. Today, spikes are typically recorded with multielectrodes, for example, 
tetrodes with four closely positioned recording contacts (\Fref{fig:Spikes:MUA-tetrode}), 
linear multielectrodes (polytrodes) with tens of contacts positioned along a straight line (\Fref{fig:Spikes:MUA-polytrode}), or 
grid-like multielectrodes with many hundred tiny contacts arranged in rectangular patterns on an electrode shaft \cite**{Jun2017}.  
On these multielectrodes, the same spike will in general show up on several contact, and a process known as 
\emph{spike sorting}~\cite**{Quiroga2007} is required to properly count spikes (and to sort recorded spikes 
into contributions from individual neurons as is often the goal). 

To develop and validate methods for spike sorting, benchmarking data where the `ground truth' is known is highly desirable. Experimental benchmarking
data is hard to come by as they require simultaneous recording of intracellular action potentials and correspond extracellular spikes.
However, model-based benchmarking data is an attractive alternative~\cite**{Einevoll2012} and the generation of such data has been pursued in several projects
\cite**{CamunasMesa2013,Hagen2015,MondragonGonzalez2017,Buccino2020}. One example is given in \Fref{fig:Spikes:MUA-tetrode} where virtual MUA signals
recorded by a tetrode is shown \cite**{Hagen2015}. The MUA signals have been computed by simulating six cortical neurons placed around a tetrode which are driven to spiking by a combination
of excitatory and inhibitory synaptic inputs (panel a). Further, noise with the same statistical properties as what is seen in real tetrode experiments, is added to the signal (panel b) 
prior to the high-paa filtering which results in the MUA benchmarking data (panel c). The final MUA data clearly shows how an action potential from a single neuron is seen
simultaneously as spikes on several of the tetrode contacts. Another example is given in \Fref{fig:Spikes:MUA-polytrode} where spikes from 16 cortical neurons are recorded by a 16-contact linear multielectode spanning the cortex. Here we observe that an action potential from a single neuron typically are observed as spikes at two to four adjacent contacts.



%%%
\begin{figure}[!ht]
\begin{center}
\includegraphics[width=0.8\textwidth]{Figures/Spikes/MUA-10}
\end{center}
\caption[]{\textbf{MUA linear multielectrode (polytrode)}
Fra Hagen 2015: Excerpts of intracellular and extracellular recordings for the 16 cells included in the example 16-channel polytrode benchmarking data set. 
(a) Somatic membrane potentials. Firing rates of cells 1-16 averaged over the 120 s real-time duration of the simulation are listed on the right hand side. (b) Superposition of extracellular potentials from all neurons and model noise, after band-pass filtering."
}
\gen{Kan vi kan legge til et panel som illustrerer polytroden?}
\label{fig:Spikes:MUA-polytrode}
\end{figure}
%%%

\subsection{\orange{Population firing-rate estimation from MUA}}  

The number of spikes that can be picked up above the noise by an electrode contact, depends on several factors. One is the volume density and morphological shapes of active
neurons around the contact. Another is the impedance and size of the contact itself. 
As discussed in \Fref{sec:Spikes:electrode_size} large electrode contacts will tend to reduce the spike amplitude through a spatial averaging effect, making it difficult to 
identify individual spikes from the recorded signal. Here an estimate of the total firing rate of the neurons surrounding the contact may be found in a different 
way,  that is, by rectification of the high-pass filtered extracellular potential~\cite**{Schroeder1998,Schroeder2001,Ulbert2001}. 

This approach was used to estimate firing rates of cortical populations of neuron in the rat somatosensory cortex based on multielectrode laminar 
recordings~\cite**{Einevoll2007,Blomquist2009}. To test the validity of the approach, a parallel modeling study was pursued where an analogous virtual experiment 
was done on a population of about 1000 layer-5 cortical neurons receiving synaptic inputs \cite**{Pettersen2008} . 



\begin{itemize}
\item Testing: \cite**{Pettersen2008,Glabska2016}
\item New figure showing all steps in process?
\end{itemize} 


%%%%
%\begin{figure}[!ht]
%\begin{center}
%\includegraphics[width=0.8\textwidth]{Figures/Spikes/MUA-2}
%\end{center}
%\caption[]{\textbf{MUA}}
%\label{Spikes:fig:MUA-A}
%\end{figure}
%%%%
%
%%%%
%\begin{figure}[!ht]
%\begin{center}
%\includegraphics[width=0.8\textwidth]{Figures/Spikes/MUA-3} \\
%\includegraphics[width=0.8\textwidth]{Figures/Spikes/MUA-4} \\
%\includegraphics[width=0.4\textwidth]{Figures/Spikes/MUA-5} 
%\end{center}
%\caption[]{\textbf{MUA}}
%\label{Spikes:fig:MUA-B}
%\end{figure}
%%%%
%
%%%%
%\begin{figure}[!ht]
%\begin{center}
%\includegraphics[width=0.4\textwidth]{Figures/Spikes/MUA-6}\\
%\includegraphics[width=0.4\textwidth]{Figures/Spikes/MUA-7}
%\end{center}
%\caption[]{\textbf{MUA}}
%\label{Spikes:fig:MUA-C}
%\end{figure}
%%%%
%
%%%%
%\begin{figure}[!ht]
%\begin{center}
%\includegraphics[width=0.8\textwidth]{Figures/Spikes/MUA-8}
%\end{center}
%\caption[]{\textbf{MUA}}
%\label{Spikes:fig:MUA-D}
%\end{figure}
%%%%
%

%\section{\red{Insights from MUA studies}} 
%\ghnote{I added this kind of subsection to most of the Part 2 - sections. I thought it might be an idea to finish the MUA, LFP, ECoG and EEG sections with summaries of what these modalities typically tell us, i.e. in terms of (i) what aspects of neural activity they reflect (spikes, synaptic inputs, dendritic ion channels, which ion channels, which kind of neurons, something on network structure, cell orientation, cortical folding etc.), and what what they can tell us about cognitive states (attentive, drowsy etc.). I am not sure about this idea, though. Maybe it will be too challenging to get an overview over the literature - we dont want to put the entire Nunez-book into the EEG-chapter.}

%%%%%%%%%%%
%% Box: Spike sorting
%%%%%%%%%%%
%%\begin{boxfloat}{Spike sorting}
%%  \label{mm:box:spike-sorting}
%\subsection{\red{Box: Spike-sorting}}
%\gen{This was a Box in the Sterratt chapter}
%%
%\centerline{\includegraphics{Figures/Spikes/Spikes-sorting-w35-r300}}\vspace*{6pt}
%%
%A sharp electrode placed in brain tissue will pick up spiking signals from several neurons. 
%However, the shapes of the spikes will be different for the different neurons, and this can be used to sort the spikes according to their neurons of origin. This is referred to as \index{spike sorting}, 
%and is a problem of great practical importance both for neuroscience research and development of neuroprosthetic devices. 
%
%In the present day with electrodes with hundreds or thousands of electrode recording contacts, fast and accurate automatic spike-sporting methods are needed to replace time-consuming manual spike-sorting methods~\cite**{Quiroga2007}. To develop and test such automatic methods, one needs
%benchmarking spiking data where the `ground truth', that is, the actual spiking times for the contributing neurons, is known~\cite**{Einevoll2012}.
%One use of the EP modelling scheme for spikes has been to generate such benchmarking data~\cite**{CamunasMesa2013,Hagen2015,MondragonGonzalez2017}. 
%
%Modern electrodes have numerous recording contacts, often placed only some micrometers apart. Thus a spike can be measured at several contacts
%simultaneously, each contact recording a slightly different shape reflecting the different positions of the contacts relative to the spiking neuron. 
%This not only allows for accurate spike sorting, but also for estimation of the spatial position of the neuron. Likewise, the spatial variation of the 
%spike shape around the neuronal soma (see Figure~\ref{Spikes:fig:MultiCompartment}) 
%depends on the details of the intracellular action potential and dendritic morphology thus also allowing for the 
%identification of neuron type~\cite**{Buccino2018}.    
%\gen{Figure to be adapted from Quiroga (2007).}
%%\end{boxfloat}
%%%%



%While the formulae above were derived for a neuron model with a single passive dendritic stick, similar expressions can be derived for 
%more complicated neuron models where several passive sticks protrude from the soma, see \citeasnoun**{Pettersen2008}.
%The main conclusions above hold also for these neuron models, in particular that spike widths always increase with distance and that
%the amplitude of a spike is proportional to $d^{k}$ where $k\sim1.5-2$. For neurons with many dendrites attached to the 
%soma, the contributions to the spike amplitude roughly adds up. A simple rule of thumb is that a neuron's spike amplitude is 
%roughly proportional to the sum of the cross-sectional areas for all dendrite branches attached directly onto to the soma. Neurons with many thick
%dendritic branches attached to the soma will thus generate the largest spikes. See~\citeasnoun**{Pettersen2008} for further discussion.
%
%
%{\bf TEXT COPIED FROM STERRATT CHAPTER ON 2020-10-13}
%
%\section{Dependence of spike size and shape on neuronal properties}
%
%Extracellular measurement of spikes from a neuron in living brains is blind in the sense that it is
%not known what type of neuron is recorded from when an electrode is lowered into the brain.
%Some neuron types produce spikes with larger amplitudes and/or broader shapes than
%others,
%%
%%\todo{TVN: Nevne at det ofte sorteres i "putative excitatory" og "putative inhibitory"?}
%%
%and as seen in Figure~\ref{mm:fig:EP-spike-MultiCompartment} both the shape and amplitude 
%depend critically on recording positions. Large spike amplitudes imply that they will be more dominant in electrical recordings,
%and ideally this bias should  be considered in the analysis of joint recordings 
%of spikes from many neurons. 
%
%To understand the link between the morphology of neurons and their spike amplitudes and shapes
%it is convenient to consider ball-and-stick neurons where a passive dendrite cable `stick' is connected to a point-like soma.
%Despite its simplicity, the ball-and-stick neuron model exhibits the key qualitative features observed in
%Figure~\ref{mm:fig:EP-spike-MultiCompartment} when the multi-compartmental EP formula in 
%Equation~\ref{mm:equation:Ve-multi-compartment}
%is used; that is, rapid attenuation of spike amplitude and increased spike width as the distance from the soma 
%increases \cite**{Pettersen2008,Pettersen2012}. 
%
%\citeasnoun**{Pettersen2008} took advantage of the mathematical tractability of the ball-and-stick model to
%derive analytical expressions for how the amplitude of the recorded spike depends on distance from the neuron as well
%as the electric properties of the neuron. The action potentials were decomposed into contributions from
%many frequency components and each frequency component was considered individually.
%This is illustrated in Figure~\ref{mm:fig:EP-spike-ball-and-stick-frequency} where the amplitudes of the different frequency components needed to
%represent the intracellular action potential (membrane potential) and extracellular spike respectively are shown.
%A key observation here is that for the extracellular spike, the largest contributions comes from frequencies larger than 100~hertz.
%%%%%%%%%%%
%% Figure: Action potential and its frequency content
%%%%%%%%%%%
%%\begin{cnfigure}{Figures/fig-not-pushed-to-github}
%%\begin{cnfigure}{Figures/mm/EP-spike-ball-and-stick-frequency-w90-r150}
%%\caption[]{
%%Frequency content of example intracellular (a) and extracellular spike (b).
%%%
%%%(a) Action potential used in simulations in Figure~\ref{mm:fig:EP-spike-ball-and-stick-results}(inset) 
%%%\todo{Comment from DCS not understood.}
%%%and its frequency content. 
%%%%The intracellular spike width is defined as the
%%%%width of the AP at half amplitude and is 0.55~ms for the standard
%%%%AP, and half the value for the narrow AP. 
%%%(b) Frequency content of example extracellular spike.
%%%Inset: Typical spike shape computed a distance $r=10~\mu$m
%%%perpendicular to the dendrite at the level of soma for a ball-and-stick 
%%%neuron with diameter $d=2~\mu$m and infinite dendrite length.
%%%Here the intracellular action potential in left panel was imposed as a voltage-clamp in the soma.
%%%The extracellular spike width is defined as the width of the negative phase at 25\% of its maximum
%%%amplitude and is 0.44~ms for the example spike.
%%%The amplitude (peak-to-peak value) of the spike is $56~\mu$V  
%%\todo{Figure + caption to be updated. Will only include standard AP in final figure.} 
%%Adapted from \citeasnoun**{Pettersen2008}.
%%}
%%\label{mm:fig:EP-spike-ball-and-stick-frequency}
%%\figpermOurs
%%\end{cnfigure}
%
%
%%\todo{Explain the idea that the question is about how currents entering the soma during the AP returns through the dendrite}
%
%\citeasnoun**{Pettersen2008} derived simple formulae relating the intracellular action potential to the extracellular spike for two limiting cases:
%when the recording is done near to the soma or far away from the soma. For recordings near the soma the 
%amplitude $|\hat{V}_\mathrm{e,near}(f,\vec{r})|$ of the spike signal for each frequency component with frequency $f$ was found to be approximated
%by
%%
%\begin{equation}
%  |\hat{V}_\mathrm{e,near}(f,\vec{r})| 
%  \propto \frac{d^{3/2}}{r} \sqrt{ \frac{f C_\mathrm{m}}{R_\mathrm{a}} }  |\hat{V}_\mathrm{s}(f)| 
%  \label{mm:equation:Ve_near}
%\end{equation}
%%
%Here $|\hat{V}_\mathrm{s}(f)|$ is the amplitude of frequency component of the soma membrane potential at the same
%frequency (see Figure~\ref{mm:fig:EP-spike-ball-and-stick-frequency}).
%For recording positions far away from the soma, the following expression was instead found:
%%%%
%\begin{equation}
%  |\hat{V}_\mathrm{e,far}(f,\vec{r})|  \propto d^{2} \frac{|\cos \theta| }{r^2  R_\mathrm{a}}  |\hat{V}_\mathrm{s}(f)| 
%  \label{mm:equation:Ve_far}
%\end{equation}
%%%%
%In these formulas, $d$ is the diameter of the dendritic
%stick, $C_\mathrm{m}$ the specific membrane capacitance, and $R_\mathrm{a}$ the specific axial resistance.  
%

%%%%%%%%%%%
%% Figure: Spike widths and amplitudes
%%%%%%%%%%%
%\begin{cnfigure}{Figures/mm/EP-spike-ball-and-stick-results-w100-r150}
%\caption[]{
%Spike widths (left) and (peak-to-peak) spike amplitudes (right) as a function of
%distance from soma for a detailed pyramidal cell model (pyramidal) and two
%types of ball-and-stick models: long, finite ball-and-stick
%model (fin.~stick, long) with diameter $d=2~\mu$m and length
%$l=1$~mm and a short, finite ball-and-stick model (fin.~stick, short) with diameter $d=1~\mu$m and length $l=0.2$~mm,
%see Figure~\ref{mm:fig:EP-spike-ball-and-stick-neuron-models}.
%The intracellular action potential shown in the inset in the left panel of 
%Figure~\ref{mm:fig:EP-spike-ball-and-stick-frequency} was imposed as a voltage-clamp in the soma.
%The EP was recorded in the
%somatic plane normal to the stick/primary apical dendrite. 
%In right panels guidelines illustrating the power-law decays $1/r$ and
%$1/r^{2}$ have been added. 
%For further details see \citeasnoun[Figure 6]{Pettersen2008}.
%\todo{Figure + caption to be updated. Adapted from \citeasnoun**{Pettersen2008}.}
%}
%\label{mm:fig:EP-spike-ball-and-stick-results}
%\figpermOurs
%\end{cnfigure}


%\paragraph{Spike amplitude dependence on distance}

%%%%%%%%%%%%%%%%%%%%%%%%%%%
%% Box: Spike sharpness
%%%%%%%%%%%    
%\begin{boxfloat}{Spike sharpness}
%  \label{mm:box:spike-sharpness}
%There are several qualitative insights regarding the sizes and widths of spikes that can 
%be found from the near-field and far-field formulae in Equations~\ref{mm:equation:Ve_near} and 
%\ref{mm:equation:Ve_far}, respectively. One relates directly to the shape of the spike:
%in the near-field expression, the high-frequency components of the spike is amplified 
%compared to the low-frequency components, that is, $\hat{V}_\mathrm{e,near}(f,r) \propto \sqrt{f}$.
%Thus close to the soma the spike is observed to be sharper than the intracellular action potential
%as observed in the insets in Figure~\ref{mm:fig:EP-spike-ball-and-stick-frequency}. 
%In the far-field regime there is no such high-frequency amplification ($\hat{V}_\mathrm{e,near}(f,r) \propto f^0 \sim 1$).
%As a consequence, spikes measured far away from the soma will have less high-frequency content than those measured close to soma.
%Thus the far-away spikes will be blunter and have larger spike widths as seen in the spike-width panel of 
%Figure~\ref{mm:fig:EP-spike-ball-and-stick-results}.
%%
%%\centerline{\includegraphics{Figures/mm/MEA-1-w43-r300}}\vspace*{6pt}
%%
%\end{boxfloat}
%%%%%%%%%%%%%%%%%%%%%%%%%%%%%%%%





%
%%%%%%%%%%%
%% Sidebox: Fourier sum
%%%%%%%%%%%
%\begin{sidebox}
%Time signals, such as the time course of a spike $V_\mathrm{e}(t)$,  can conveniently 
%be represented as a sum of \firstterm{Fourier components} with different frequencies $f$. 
%Such a Fourier sum can be constructed in
%various ways. The derivations in 
%Section~\ref{mm:sec:spike-widths-amplitudes}, building on \citeasnoun**{Pettersen2008}, use the  
%convention that a time signal $S(t)$ is the real part of the complex sum $\sum_{f}  \hat{S}(f) \exp (j 2 \pi f t)$. 
%Here $j$ is the unit of imaginary numbers, and  $\hat{S}(f)$ is in general a complex number. 
%\end{sidebox}
%%
%
%%%%%%%%%%%
%% Figure: Spike widths and amplitudes
%%%%%%%%%%%
%\begin{cnfigure}{Figures/mm/EP-spike-ball-and-stick-results-w100-r150}
%\caption[]{
%Spike widths (left) and (peak-to-peak) spike amplitudes (right) as a function of
%distance from soma for a detailed pyramidal cell model (pyramidal) and two
%types of ball-and-stick models: long, finite ball-and-stick
%model (fin.~stick, long) with diameter $d=2~\mu$m and length
%$l=1$~mm and a short, finite ball-and-stick model (fin.~stick, short) with diameter $d=1~\mu$m and length $l=0.2$~mm,
%see Figure~\ref{mm:fig:EP-spike-ball-and-stick-neuron-models}.
%The intracellular action potential shown in the inset in the left panel of 
%Figure~\ref{mm:fig:EP-spike-ball-and-stick-frequency} was imposed as a voltage-clamp in the soma.
%The EP was recorded in the
%somatic plane normal to the stick/primary apical dendrite. 
%In right panels guidelines illustrating the power-law decays $1/r$ and
%$1/r^{2}$ have been added. 
%For further details see \citeasnoun[Figure 6]{Pettersen2008}.
%\todo{Figure + caption to be updated. Adapted from \citeasnoun**{Pettersen2008}.}
%}
%\label{mm:fig:EP-spike-ball-and-stick-results}
%\figpermOurs
%\end{cnfigure}
%
%%%%%%%%%%%
%% Figure: Frequency-dependent distribution of return currents
%%%%%%%%%%%
%\begin{cnfigure}{Figures/mm/EP-spike-ball-and-stick-sketch-w70-r300}
%%\begin{cnfigure}{Figures/fig-not-pushed-to-github}
%\caption[]{Illustration of ball-and-stick neuron and its frequency-dependent 
%distribution of dendritic return currents following injection of a sinusoidal current into the soma.
%The net current entering the soma will enter the dendrite as an axial current, and return to the 
%ECS via the dendrite membrane. The inset shows the spatial distribution of this return current
%for different frequencies. The higher the frequency, the closer the return currents will be and
%the smaller the frequency-dependent length constant $\hat{\lambda}(f)$, reflecting the weighted mean 
%of the return-current positions (see sidebox), will be. 
%\todo{Figure + caption to be updated. Adapted from \citeasnoun**{Pettersen2012}.}
%\todo{June 2019: Move inbset to Ch. 5?}
%}
%\label{mm:fig:EP-spike-ball-and-stick-sketch}
%\figpermOurs
%\end{cnfigure}
%%
%
%
%
%%%%%%%%%%%
%% Box: Ball-and-stick model for spikes
%%%%%%%%%%%
%\begin{boxfloat}{Ball-and-stick model for spikes}
%  \label{mm:box:ball-and-stick-spikes}  
%To understand the dependence of spike shapes and amplitudes on model parameters for the ball-and-stick neuron, it is useful
%to consider the EP set up by each of the frequency (Fourier) components of the action potentials separately. 
%For recording positions $\vec{r}$ very close to the soma, the contribution from the soma current will dominate over the contributions from the dendrite 
%in the sum giving the EP in Equation~\ref{mm:equation:Ve-multi-compartment}. Then the amplitude of the
%predicted oscillating EP $|\hat{V}_\mathrm{e,near}(f,\vec{r})|$ is approximately given by 
%%
%\begin{equation}
%  |\hat{V}_\mathrm{e,near}(f,\vec{r})| = \frac{|\hat{I}_\mathrm{s}(f)|}{4 \pi \sigma r} 
%  \label{mm:box:equation:Ve_near_1}
%\end{equation}
%%
%where $|\hat{I}_\mathrm{s}(f)|$ is the amplitude of the oscillating current through the soma membrane.
%%and also the axial current entering the dendrite from the soma compartment. 
%For the relatively high frequencies of most relevance for the spike, this soma current is related to the soma membrane potential 
%$\hat{V}_\mathrm{s}(f)$ through~\cite**{Pettersen2008}
%%
%\begin{equation}
%  |\hat{I}_\mathrm{s}(f)| =  \frac{\pi^{3/2} d^{3/2}}{\sqrt{2}} \sqrt{ \frac{f C_\mathrm{m}}{R_\mathrm{a}} }  |\hat{V}_\mathrm{s}(f)|
%  \label{mm:box:equation:Isoma}
%\end{equation}
%%
%and the spike EP is thus found to be
%%
%\begin{equation}
%  |\hat{V}_\mathrm{e,near}(f,\vec{r})| 
%  = \frac{\sqrt{\pi}}{4 \sqrt{2} \sigma}
%     \frac{d^{3/2}}{r} 
%     \sqrt{ \frac{f C_\mathrm{m}}{R_\mathrm{a}} }  |\hat{V}_\mathrm{s}(f)| 
%  \propto \frac{d^{3/2}}{r} \sqrt{ \frac{f C_\mathrm{m}}{R_\mathrm{a}} }  |\hat{V}_\mathrm{s}(f)| 
%  \label{mm:box:equation:Ve_near_2}
%\end{equation}
%%
%\todo{June 2019: Coordinated with or moved to Ch. 5.}
%%
%For recording positions further away from the soma, the contribution from return membrane currents must be taken into
%account in the sum in Equation~\ref{mm:equation:Ve-multi-compartment}. An approximate way of doing this is to
%assume all return currents to leave the dendrite at a single height $\lambda_\mathrm{AC}(f)$ above the soma, where 
%this frequency-dependent length constant corresponds to the weighted mean of the positions of the return currents along
%the dendrite stick. (The subscript `AC' denotes `alternating current'.)
%Then the EP can be approximated by using the 
%dipolar expression in Equation~\ref{mm:equation:Ve-dipole-p}, that is,
%%%%
%\begin{equation}
%  |\hat{V}_\mathrm{e,far}(f,\vec{r})| =  \frac{|p(f) \cos \theta|}{4 \pi \sigma r^2} 
%                                            = \frac{| \hat{I}_{s}(f) \lambda_\mathrm{AC}(f) \cos \theta|}{4 \pi \sigma r^2}   
%                                                                                        \label{mm:box:equation:Ve_far_1}
%\end{equation}
%%%%
%One way to define an AC length constant is as the mean value of the 
%envelope of the sinusoidally varying (normalized) membrane current
%$\hat{i}_\mathrm{m}$ weighted with distance $z$ from soma, 
%see Figure~\ref{mm:fig:EP-spike-ball-and-stick-sketch}. 
%For an infinite dendrite stick this corresponds to
%%
%\begin{equation}
%  \lambda_\mathrm{AC}^\infty(f) = \frac{\int_0^\infty z |\hat{i}_\mathrm{m}| dz}{\int_0^\infty |\hat{i}_\mathrm{m}| dz} 
%\nonumber
%%=  \frac{\sqrt{2}\lambda}{\sqrt{\sqrt{W^2+1}+1}}.
%\end{equation}
%%
%
%For high frequencies ($f \gg 1/2 \pi R_\mathrm{m} C_\mathrm{m}$) this is after some algebra 
%found to give (see \citeasnoun[Appendix C]{Pettersen2008} for details)
%%
%\begin{equation}
% \lambda_\mathrm{AC}^\infty(f) =  \frac{\lambda}{\sqrt{\pi f \tau}} = 
%  \frac{1}{2\sqrt{\pi}} \sqrt{\frac{d}{f R_\mathrm{m} C_\mathrm{m}}}
%\label{mm:box:equation:approx_lambda_ac}
%\end{equation}
%%
%where $\lambda$ is the cable length constant from 
%%Chapter~\ref{XX:chap:XX}
%Chapter~5. f{REF} 
%
%%










