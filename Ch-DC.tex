\chapter{Local Field Potentials: Non-Synaptic contributions}
\label{chap:DC}
\ghnote{Som diskutert paa moete kommer et kapittel om non-synaptic LFP. 
Snakket med Klas, og kom fram til at raadende tittelforslag. Tittelen er LFP, som vi kan definere
som ekstracellulaert potensial der alt over en viss grense (typisk 300 eller 500 Hz) er filtrert bort
for aa fjerne spikes. Mye av det vi ser i LFP skyldes synapseinput, men noe skyldes andre ting. 

Hvordan definere synaptic? 
Alt som trigges av synaptisk input? Dette er jo det meste, inkludert spikes. 
Er "spikes" en del av synaptic LFP? 
Eller er synaptic LFP kun synapsestroem + returstroemmer av passiv og kapasitiv natur?

Da vi ble enige om dette kapittelet, snakket vi i hovedsak om denne DC-komponenten, modellert av Torbjorn. 
Men her passer det vel ogsaa aa ta opp sanne ting som om evt. spikes, eller saerlig calcium spikes bidrar til LFP?
Noen nevroner er jo dessuten spontant aktive - dvs. kan lage spikes og greier uten at de faar synaptisk input. Hva med dem?

Er tittelen dum? Hvis vi har definert LFP som alt over cutoff 0.1 Hz, saa hoerer ikke DC-komponenten hjemme her...
Er cutoff en del av LFP definisjonen, eller er den bare en teknikalitet? Hvordan ser vi paa dette?
}


\section{\red{Contributions from spikes?}}
\ghnote{Har ikke TVN simulert LFP fra Ca-spike? 
Regner med at Ca-spikes er godt synlig i LFP. I hvilken grad skal vi snakke om dette som
synaptic return-current, vs. active spike?}


\section{\red{DC-components from resting neurons}}
\ghnote{TVN viser figuren sin.}


\section{\red{Contributions of non-neuronal origin}}
\ghnote{Nevne diffusjonspotensialer her, og referere videre.}
\ghnote{Har ogsaa hoert at svette og bevegelsesartefakter bidrar i de lavfrekvente omraadene.}