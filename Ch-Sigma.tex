\section{Extracellular conductivity}
\label{sec:Sigma}
\index{Extracellular conductivity}

\begin{itemize}
\item Experimental measurements \citep{Miceli2017}
\item Theoretical explorations \citep{Meffin2012,Tahayori2012,Meffin2014,Tahayori2014}
\end{itemize}

\begin{figure}[!ht]
\begin{center}
\includegraphics[width=0.6\textwidth]{Figures/Sigma/frequency_dependence.png}
\end{center}
\caption{\textbf{Literature review of reported conductivities in various species and experimental setups.} 
Most studies seem to indicate a very weak frequency dependence of the extracellular conductivity\index{conductivity}, which would have a negligible effect on measured extracellular potentials \citep{Miceli2017}. The very low and strongly frequency dependent values measured by \cite{Gabriel1996} represents an outlier, and although it has received substantial attention, it has to the best of our knowledge not been reproduced by any other study.
For details about the data, see \cite{Miceli2017}, and references therein \citep{Ranck1963, Gabriel1996, Logothetis2007, Elbohouty2013, Wagner2014}
}
\label{Sigma:fig:freq_dep}
\end{figure}


\begin{figure}[!ht]
\begin{center}
\includegraphics[width=0.6\textwidth]{Figures/Sigma/resistivity_maxwell.png}
\end{center}
\caption{\textbf{From Nunez} \tvnnote{Ta med noe slikt?}}
\label{Sigma:fig:maxwell_resistivity}
\end{figure}

\subsection{Capacitive effects in neuronal tissue}
At high frequencies, extracellular currents could pass through neurons as capacitive currents.
\tvnnote{Utledning tilsvarende Appendix B i Nunez?}