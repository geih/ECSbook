\chapter{TVN/GH: MEG}
\label{chap:MEG}
\index{MEG}
\ghnote{
Here we must understand the relationship between "impressed currents" and "primary currents" as they are used in the MEG litterature, i.e., in the book Brain Signals. So far, we have MEG only as an application-chapter. Should we have a theory chapter magnetic fields it in Part 1, or will we sneak the theory in here as we go along?}

\snnote{I've looked into this difference in notation, and explained my findings below.=)}
\snnote{I ended up spending way too much time on magnetic field theory, so I thought I might as well write down the derivation of the Amp\`ere-Laplace law, and some notes on contributions from the primary currents and volume currents. Don't know if this is the place for this theory, or if any of it should be included at all, so feel free to rewrite, move and delete it as you like, just thought I'd write it down just in case.}
\snnote{If we end up using the term "primary current" ${\bf i}_\mathrm{p}$, we need a different notation, as p means dipole. Most magnetic people use ${\bf J}_\mathrm{p}$, some use ${\bf J}_\mathrm{i}$..}
\sntxt{
The difference between our notation and the \cite**{Ilmoniemi2019} notation, is that they include the impressed current sources (which they call primary currents, i.e. transmembrane currents) in the total current. We, on the other hand, keep a strict distinction between impressed currents and the resulting tissue current, by keeping the impressed current in the current source density term $C$.

I'll explain the difference by outlining how we and \cite{Ilmoniemi2019} derive the Poisson equation:

We write the continuity equation as follows:

\begin{equation} \label{eq:MEG:continuity_1}
{\bf \nabla} {\bf i}_t = C
\end{equation}
based on \citeasnoun**{Gratiy2017}.

${\bf i}_\mathrm{t}$ is here the tissue current density resulting from the current source density $C$. All "impressed currents" (i.e. transmembrane currents), are in our case included in $C$.

Next, we write Ohm's law this way:

\begin{equation}\label{eq:MEG:ohm_1}
{\bf i}_\mathrm{t} = \sigma {\bf E}.
\end{equation}

Inserting Equation \eqref{eq:MEG:ohm_1} into Equation \eqref{eq:MEG:continuity_1}, applying ${\bf E} = -{\bf \nabla} V_\mathrm{e}$, gives the Poisson equation:

\begin{equation}\label{eq:MEG:poisson1}
{\bf \nabla} \cdot (\sigma {\bf \nabla} V_\mathrm{e}) = -C.
\end{equation}



In \citeasnoun**{Ilmoniemi2019},\citeasnoun**{Sarvas1987} and \citeasnoun**{Hamalainen1993} they write the continuity equation as

\begin{equation} \label{eq:MEG:continuity_2}
{\bf \nabla} \cdot {\bf i} = 0
\end{equation}

Where {\bf i} is the total current density:
\begin{equation}\label{eq:MEG:ohm_2}
{\bf i} = {\bf i}_\mathrm{p} - \sigma {\bf \nabla} V_\mathrm{e}.
\end{equation}
Here ${\bf i}_\mathrm{p}$ is referred to as the \textit{source current density}/ \textit{primary current density}.  All current sources, i.e. transmembrane currents, that we have knowledge of are included in ${\bf i}_\mathrm{p}$, so we could as well have named ${\bf i}_\mathrm{p}$ the "impressed current density". In terms of \Fref{eq:VC:Sergey4}, ${\bf i}_\mathrm{p}$ would be the same as $\left< {\bf i}_\mathrm{tot} \right>_c$, given that we had modeled all membrane source currents.

The second term on the right hand side of Equation \eqref{eq:MEG:ohm_2} $- \sigma {\bf \nabla} V_\mathrm{e}$ is the ohmic/ volume current density generated by the primary/ impressed current density ${\bf i}_\mathrm{p}$. This volume current density is the same as our tissue current density.

The Poisson equation can further be derived by inserting Equation \eqref{eq:MEG:ohm_2} into \eqref{eq:MEG:continuity_2}, such that:

\begin{equation}\label{eq:MEG:poisson2}
{\bf \nabla} \cdot (\sigma {\bf \nabla} V_\mathrm{e}) = {\bf \nabla} \cdot {\bf i}_\mathrm{p}.
\end{equation}


Comparing Equation \eqref{eq:MEG:poisson1} and \eqref{eq:MEG:poisson2}, we can see that the relation between the primary current density and the current source density is the following:

\begin{equation}
C = - {\bf \nabla} \cdot {\bf i}_\mathrm{p}.
\end{equation}

So, as we typically write the solution of the Poisson equation as

\begin{equation}
V_\mathrm{e}({\bf r}) = \frac{1}{4\pi \sigma_\mathrm{t}} \iiint_\Omega \frac{C({\bf r}')}{|{\bf r} - {\bf r}'|} d\Omega' 
\end{equation}

\citeasnoun**{Ilmoniemi2019} expresses the extracellular potential without reference to the current source density:

\begin{equation}
V_\mathrm{e}({\bf r}) = - \frac{1}{4\pi \sigma_\mathrm{t}} \iiint_\Omega \frac{{\bf \nabla}' \cdot {\bf i}_\mathrm{p}({\bf r}')}{|{\bf r} - {\bf r}'|} d\Omega'.
\end{equation}
}


\snnote{If we're going to include the full derivation of the Biot-Savart law, we could do something along these lines:}

\slntxt{
The quasistatic version of Maxwell's fourth equation explains how an electric current gives rise to a magnetic field {\bf B}. In the brain, the magnetic permeability $\mu$ is very close to the magnetic permeability in vacuum $\mu_0$ \cite{Hamalainen}, such that:

\begin{equation}\label{eq:MW4_qs_2}
\nabla \times {\bf B} = \mu_0 {\bf i}
\end{equation}
where ${\bf i}$ is the total current density produced by neural activity.

In order to derive an expression for ${\bf B}$ (as in \cite{Griffiths1999}), we start by defining the vector potential ${\bf A}$

\begin{equation}\label{eq:defA}
{\bf B} = {\bf \nabla} \times {\bf A},
\end{equation}

with zero divergence ${\bf \nabla \cdot A} = 0$. Inserting \eqref{eq:defA} into \eqref{eq:MW4_qs_2}, we see that:

\begin{equation*}
\nabla \times {\bf B} = \nabla \times ({\bf \nabla} \times {\bf A})
					 = {\bf \nabla} ({\bf \nabla} \cdot {\bf A})
					    - {\bf \nabla}^2 {\bf A}
					 = \mu_0 {\bf i}
\end{equation*}

Since we have defined ${\bf \nabla} \cdot {\bf A} = 0$ (see \cite{Griffiths1999}), we end up with the Poisson equation:

\begin{equation}\label{eq:poisson_A}
{\bf \nabla}^2 {\bf A} = -\mu_0 {\bf i},
\end{equation}

which can be solved in the same way as \sntxt{the electric field Poisson equation}, assuming that ${\bf i}$ goes to zero at infinity:

\begin{equation}\label{eq:A}
{\bf A}({\bf r}) = \frac{\mu_0}{4\pi} \int \frac{{\bf i}({\bf r}')}{|{\bf r} - {\bf r}'|} dV'.
\end{equation}

We now obtain the following expression for the magnetic field:

\begin{equation}\label{eq:MEG:B}
{\bf B}({\bf r}) = \frac{\mu_0}{4\pi} \int {\bf \nabla} \times \frac{ {\bf i}({\bf r}')}{|{\bf r} - {\bf r}'|} dV'.
\end{equation}

We can now apply the three tricks for rewriting \eqref{eq:MEG:B}:

(i) the cross product identity
\begin{equation*}
{\bf \nabla} \times \left(\frac{{\bf i}({\bf r}')}{|{\bf r} - {\bf r}'|}\right) = \frac{1}{|{\bf r} - {\bf r}'|} {\bf \nabla} \times {\bf i}({\bf r}') - {\bf i}({\bf r}') \times {\bf \nabla} \left(\frac{1}{|{\bf r} - {\bf r}'|}\right),
\end{equation*}

(ii) ${\bf \nabla} \times {\bf i}({\bf r}') = 0$ since ${\bf i}({\bf r}'$ is independent of ${\bf r}$, and 

(iii) ${\bf \nabla} 1/|{\bf r} - {\bf r}'| = -({\bf r} - {\bf r}')/|{\bf r} - {\bf r}'|^3$

and end up with the Amp\`ere-Laplace law:

\begin{equation}\label{eq:MEG:ampere-laplace}
{\bf B}({\bf r}) = \frac{\mu_0}{4\pi} \int \frac{ {\bf i}({\bf r}') \times ({\bf r} - {\bf r}')}{|{\bf r} - {\bf r}'|^3} dV'.
\end{equation}
}
\snnote{Deriving the Amp\`ere-Laplace law is not strictly necessary, so we could just start from Amp\`ere-Laplace, and take it from there..}

\slntxt{
From Amp\`eres circuit law (\Fref{eq:Basics:Max4}), the Amp\`ere-Laplace law can be derived:

\begin{equation}\label{eq:MEG:ampere-laplace}
{\bf B}({\bf r}) = \frac{\mu_0}{4\pi} \int \frac{ {\bf i}({\bf r}') \times ({\bf r} - {\bf r}')}{|{\bf r} - {\bf r}'|^3} dV',
\end{equation}

giving the magnetic field induced by a current ${\bf i}$.

In order to take a closer look at the magnetic field contributions from primary currents and volume currents, we rewrite \Fref{eq:MEG:ampere-laplace}, by the three following steps:

(i) inserting the identity $({\bf r} - {\bf r}')/|{\bf r} - {\bf r}'|^3 = {\bf \nabla}' 1/|{\bf r} - {\bf r}'|$,

(ii) and the cross product rule, to see that
${\bf i}({\bf r}') \times {\bf \nabla}' (1/|{\bf r} - {\bf r}'|) = ({\bf \nabla}' \times {\bf i}({\bf r}')/|{\bf r} - {\bf r}'| - {\bf \nabla}' \times ({\bf i}({\bf r}'))/|{\bf r} - {\bf r}'|)$

(iii) before applying Stoke's theorem $-\int {\bf \nabla}' \times ({\bf i}({\bf r}')/|{\bf r} - {\bf r}'|)dV = \int {\bf i}({\bf r}')/|{\bf r} - {\bf r}'| \times d{\bf S}$, which is zero if ${\bf i}({\bf r}')$ goes to zero sufficiently fast, when ${\bf r} \to \infty$, we end up with:

\begin{equation}\label{eq:MEG:ampere-laplace2}
{\bf B}({\bf r}) = \frac{\mu_0}{4\pi} \int \frac{ {\bf \nabla}' \times {\bf i}({\bf r}')}{|{\bf r} - {\bf r}'|} dV'.
\end{equation}

Here, the current density ${\bf i}$ produced by neural activity can be written as the sum of the primary current density and the resulting volume current density ${\bf i} = {\bf i}_p - \sigma {\bf \nabla} V$, giving

\begin{equation}\label{eq:MEG:ampere-laplace2}
{\bf B}({\bf r}) = \frac{\mu_0}{4\pi} \int \frac{ {\bf \nabla}' \times {\bf i}_\mathrm{p}({\bf r}')}{|{\bf r} - {\bf r}'|} dV' - \frac{\mu_0}{4\pi} \int \frac{ {\bf \nabla}' \times \sigma {\bf \nabla}' V}{|{\bf r} - {\bf r}'|} dV'.
\end{equation}

Finally, we use the calculus steps described above in reverese order on the first term, and the relation ${\bf \nabla}' \times \sigma {\bf \nabla}' V = {\bf \nabla}'\sigma \times {\bf \nabla}' V$ (since ${\bf \nabla}' \times {\bf \nabla}' V = 0)$, such that


\begin{equation}\label{eq:MEG:ampere-laplace-split}
{\bf B}({\bf r}) = {\bf B}_\mathrm{0} - \frac{\mu_0}{4\pi} \int \frac{{\bf \nabla}' \sigma \times {\bf \nabla}' V}{|{\bf r} - {\bf r}'|} dV',
\end{equation}

where 

\begin{equation}\label{eq:MEG:B0}
{\bf B}_0({\bf r}) = \frac{\mu_0}{4\pi} \int \frac{ {\bf i}_\mathrm{p}({\bf r}') \times ({\bf r} - {\bf r}')}{|{\bf r} - {\bf r}'|^3} dV'.
\end{equation}

In an infinite, homogeneous conductor ${\bf \nabla}'\sigma = 0$, and we can write ${\bf B} = {\bf B}_\mathrm{0}$, meaning that we don't need to consider volume currents. In this case, it can be useful to express the magnetic field as a function of the current dipole moment. The current density ${\bf i}_\mathrm{p}$ from a current dipole ${\bf p}$ at position ${\bf r}_\mathrm{p}$ can be expressed as

\begin{equation}\label{eq:MEG:i_dipole}
{\bf i}_\mathrm{p} = {\bf p}({\bf r}_\mathrm{p}) \delta ({\bf r} - {\bf r}_\mathrm{p}).
\end{equation}

Inserting \Fref{eq:MEG:i_dipole} into \Fref{eq:MEG:B0}, we obtain

\begin{align}\label{eq:MEG:biot-savart}
{\bf B}_\mathrm{p} &= \frac{\mu_0}{4 \pi} \int \frac{{\bf p}({\bf r}_\mathrm{p}) \times ({\bf r} - {\bf r}_\mathrm{p}) \delta ({\bf r} - {\bf r}_\mathrm{p})}{|{\bf r}_\mathrm{p}|^3} \nonumber \\
&= \frac{\mu_0}{4 \pi} \frac{{\bf p}({\bf r}_\mathrm{p}) \times ({\bf r} - {\bf r}_\mathrm{p})}{|{\bf r} - {\bf r}_\mathrm{p}|^3}.
\end{align}


}

\sntxt{

\begin{itemize}
\item A radial dipole in a spherical volume conductor will give a zero contribution to the magnetic field outside of the sphere. \cite**{Katila1983}, \cite**{Sarvas1987} and \cite**{Hamalainen1993}.
\item A current dipole in a spherical head model will give a zero contribution to the radial component of the magnetic field \cite**{Hamalainen1993}.
\item \citeasnoun**{VanUitert2003} have used FEM to compute ${\bf B}$ with and without volume current contributions from randomly placed and oriented dipoles in a sphere. The relative RMS error for computations without volume currents (compared to analytical calculations) was $0.044 +- 0.042$. (While the relative RMS error for FEM-computed B-field with volume currents was $0.012 +- 0.014$). See their Figure 3. 
\item \citeasnoun**{Geselowitz1970} derives ${\bf H}$ from Maxwell for primary and volume currents. (Magnetic dipole moment is independent of chosen origin.)
\item \citeasnoun**{Nunez2006} page 127: "At low frequencies in the brain, electric and magnetic fields are uncoupled. Thus magnetic fields are due only to currents and may be calculated from the Biot-Savart law (\fref{eq:MEG:biot-savart}) [...] It is assumed in \fref{eq:MEG:biot-savart} that" $R >> d$. They don't give any other explanation of this. I guess that what they're thinking of is that the potential from a dipole  $V~1/R^2$ and the contribution to ${\bf B}$ from volume currents will fall as $B_{volume} ~ 1/R^4$. 
\item Further in \citeasnoun**{VanUitert2003} they find that in a realistic head model, the magnetic field contributions from volume currents are of the same magnitude as primary current contributions.
\end{itemize}
}

\section{\red{TVN: Insights from MEG studies} }
The human being is essentially just a very weak electromagnet. 