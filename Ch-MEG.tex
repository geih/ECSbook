\chapter{TVN/GH: MEG}
\label{chap:MEG}
\index{MEG}
\ghnote{
Here we must understand the relationship between "impressed currents" and "primary currents" as they are used in the MEG litterature, i.e., in the book Brain Signals. So far, we have MEG only as an application-chapter. Should we have a theory chapter magnetic fields it in Part 1, or will we sneak the theory in here as we go along?}

\snnote{I've looked into this difference in notation, and explained my findings below.=) When it comes to magnetic field theory, I don't see how we can derive B without reference to the primary current.. }
\sntxt{
The difference between our notation and the \cite**{Ilmoniemi2019} notation, is that they include the impressed current sources (which they call primary currents, i.e. transmembrane currents) in the total current. We, on the other hand, keep a strict distinction between impressed currents and the resulting tissue current, by keeping the impressed current in the current source density term $C$.

I'll explain the difference by outlining how we and \cite{Ilmoniemi2019} derive the Poisson equation:

We write the continuity equation as follows:

\begin{equation} \label{eq:MEG:continuity_1}
{\bf \nabla} {\bf i}_t = C
\end{equation}
based on \citeasnoun**{Gratiy2017}.

${\bf i}_\mathrm{t}$ is here the tissue current density resulting from the current source density $C$. All "impressed currents" (i.e. transmembrane currents), are in our case included in $C$.

Next, we write Ohm's law this way:

\begin{equation}\label{eq:MEG:ohm_1}
{\bf i}_\mathrm{t} = \sigma {\bf E}.
\end{equation}

Inserting Equation \eqref{eq:MEG:ohm_1} into Equation \eqref{eq:MEG:continuity_1}, applying ${\bf E} = -{\bf \nabla} V_\mathrm{e}$, gives the Poisson equation:

\begin{equation}\label{eq:MEG:poisson1}
{\bf \nabla} \cdot (\sigma {\bf \nabla} V_\mathrm{e}) = -C.
\end{equation}



In \citeasnoun**{Ilmoniemi2019},\citeasnoun**{Sarvas1987} and \citeasnoun**{Hamalainen1993} they write the continuity equation as

\begin{equation} \label{eq:MEG:continuity_2}
{\bf \nabla} \cdot {\bf i} = 0
\end{equation}

Where {\bf i} is the total current density:
\begin{equation}\label{eq:MEG:ohm_2}
{\bf i} = {\bf i}_\mathrm{p} - \sigma {\bf \nabla} V_\mathrm{e}.
\end{equation}
Here ${\bf i}_\mathrm{p}$ is referred to as the \textit{source current density}/ \textit{primary current density}.  All current sources, i.e. transmembrane currents are included in ${\bf i}_\mathrm{p}$, so we could as well have named ${\bf i}_\mathrm{p}$ the "impressed current density".
The second term on the right of Equation \eqref{eq:MEG:ohm_2} $- \sigma {\bf \nabla} V_\mathrm{e}$ is the ohmic/ volume current density generated by the primary/ impressed current density ${\bf i}_\mathrm{p}$.

The Poisson equation can further be derived by inserting Equation \eqref{eq:MEG:ohm_2} into \eqref{eq:MEG:continuity_2}, such that:

\begin{equation}\label{eq:MEG:poisson2}
{\bf \nabla} \cdot (\sigma {\bf \nabla} V_\mathrm{e}) = {\bf \nabla} \cdot {\bf i}_\mathrm{p}.
\end{equation}


Comparing Equation \eqref{eq:MEG:poisson1} and \eqref{eq:MEG:poisson2}, we can see that the relation between the primary current density and the current source density is the following:

\begin{equation}
C = - {\bf \nabla} \cdot {\bf i}_\mathrm{p}.
\end{equation}

So, as we typically write the solution of the Poisson equation as

\begin{equation}
V_\mathrm{e}({\bf r}) = \frac{1}{4\pi \sigma_\mathrm{t}} \iiint_\Omega \frac{C({\bf r}')}{|{\bf r} - {\bf r}'|} d\Omega' 
\end{equation}

\citeasnoun**{Ilmoniemi2019} expresses the extracellular potential without reference to the current source density:

\begin{equation}
V_\mathrm{e}({\bf r}) = - \frac{1}{4\pi \sigma_\mathrm{t}} \iiint_\Omega \frac{{\bf \nabla}' \cdot {\bf i}_\mathrm{p}({\bf r}')}{|{\bf r} - {\bf r}'|} d\Omega'.
\end{equation}
}




%The quasistatic version of Maxwell's fourth equation explains how an electric current gives rise to a magnetic field {\bf B}. In the brain, the magnetic permeability $\mu$ is very close to the magnetic permeability in vacuum $\mu_0$ \cite{Hamalainen}, such that:
%
%\begin{equation}\label{eq:MW4_qs_2}
%\nabla \times {\bf B} = \mu_0 {\bf i}
%\end{equation}
%
%In order to derive an expression for ${\bf B}$ (as in \cite{Griffiths1999}), we start by defining the vector potential ${\bf A}$
%
%\begin{equation}\label{eq:defA}
%{\bf B} = {\bf \nabla} \times {\bf A},
%\end{equation}
%
%with zero divergence ${\bf \nabla \cdot A} = 0$. Inserting \eqref{eq:defA} into \eqref{eq:MW4_qs_2}, we see that:
%
%\begin{equation*}
%\nabla \times {\bf B} = \nabla \times ({\bf \nabla} \times {\bf A})
%					 = {\bf \nabla} ({\bf \nabla} \cdot {\bf A})
%					    - {\bf \nabla}^2 {\bf A}
%					 = \mu_0 {\bf i}
%\end{equation*}
%
%Since we have defined ${\bf \nabla} \cdot {\bf A} = 0$ (see \cite{Griffiths1999}), we end up with the Poisson equation:
%
%\begin{equation}\label{eq:poisson_A}
%{\bf \nabla}^2 {\bf A} = -\mu_0 {\bf i},
%\end{equation}
%
%which can be solved in the same way as \sntxt{the electric field Poisson equation}, assuming that {\bf i} goes to zero at infinity:
%
%\begin{equation}\label{eq:A}
%{\bf A}({\bf r}) = \frac{\mu_0}{4\pi} \int \frac{{\bf i}({\bf r}')}{|{\bf r} - {\bf r}'|} dV'.
%\end{equation}
%
%We now obtain the following expression for the magnetic field:
%
%\begin{equation}\label{eq:B}
%{\bf B}({\bf r}) = \frac{\mu_0}{4\pi} \int {\bf \nabla} \times \frac{ {\bf i}({\bf r}')}{|{\bf r} - {\bf r}'|} dV'.
%\end{equation}
%
%Using the identity
%
%${\bf i} = {\bf i}_p - \sigma {\bf \nabla} V$

\section{\red{TVN: Insights from MEG studies} }
The human being is essentially just a very weak electromagnet. 