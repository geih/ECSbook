\section{Schemes for computing EPs from neural activity}
\label{sec:schemes}
\ghnote{Geir and Torbjorn will write most this. Torbjorn takes care of the LFPy-part, Geir takes care of the other schemes.}
\ghnote{GH: I reorganized this following the logic from the theory part - starting with the VC-stuff.}

We may classify the schemes for computing extracellular dynamics by whether or not they are self consistent and whether or not they include ion concentration dynamics (Fig. \ref{Schemes:fig:schemes}). By self consistent, we mean that the neuronal and extracellular dynamics are modelled on a unified framework, so that (ephaptic) effects of extracellular variables on the neurodynamics are accounted for. 


\begin{figure}[!ht]
\begin{center}
\includegraphics[width=0.8\textwidth]{Figures/Schemes/schemes.png}
\end{center}
\caption{\textbf{Overview of schemes for computing extracellular dynamics.}}
\label{Schemes:fig:schemes}
\end{figure}

\subsection{Schemes based on standard VC-theory}
Based on standard VC theory - compute neurodynamics first, and then have an analytical solution for the ECS-potential.
Not self consistent, no ephaptic effects. Assumes constant ion concentrations.

Present LFPy here. Explain that there are different ways of simulating the neurodynamics, and then enter subsubsections: 

\subsubsection{Neurodynamics based on multicompartmental neuron models}
The standard way - what LFPy was originally designed for \cite{Hagen2018}.
Same trick used earlier \citep{Holt1999}.

\subsubsection{Neurodynamics from point-neuron models}
Point neuron models do not generate extracellular fields. Sad, because simulations would be much faster if we could use point neuron models. Trick to do this, Hybrid LFPy \citep{Hagen2016}, Skaar et al (in revision)

\subsubsection{Neurodynamics using firing-rate models}
Would make things even faster. Population firing-rate models  \citep{Hagen2016}. Kernel trick (Ness et al, on-going project) 


\subsection{Extracellular Kirchhoff-Nernst-Planck (KNP) scheme}
As in \cite{Solbra2018}. Keeps track of ion concentrations and accounts for diffusive effects on ECS potentials. Theory presented in Chapter \ref{sec:eldiff}. Perhaps it is ok to have it both places. 


\subsection{The Extracellular-Membrane-Intracellular (EMI) scheme}
Self consistent scheme, assuming constant ion concentration, includes only drift currents. Accounts for ephaptic effects \cite{Tveito2019}. More complete than LFPy, but has the disadvantage that it is computationally expensive. Simulations based on FEM.


\subsection{Self consistent electrodiffusive schemes}
Keep track of all variables ($c_k$ and $\phi$) in all compartments. Diffusion and drift included. 

\subsubsection{The KNP-EMI-scheme}
Already presented briefly in chapter \ref{sec:eldiff}. Perhaps remove it from there and put it only here. 

\subsubsection{The PNP-scheme}
Already presented briefly in chapter \ref{sec:eldiff}. Perhaps remove it from there and put it only here. 

\subsubsection{Domain models}
Perhaps have a little chapter about the Mori domain models, although they are essentially the KNP-EMI-scheme but without explicit geometries. I am not sure that they are meaningfull in terms of predicting extracellular potentials though - at least not others than the low frequency (DC-like) part of them during spreading depression. Then again, in terms of predicting extracellular potentials on the large coarse grained scale the, also the PNP and KNP-EMI schemes probably come short, as they can not be "coarse-grained" and are to computationally heavy to run for any large scale system. 



